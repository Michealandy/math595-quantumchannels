\documentclass[../../note.tex]{subfiles}

\begin{document}

\chapter{Quantum Optics by Carlos}
\section{Prerequisites}
This section is used to collect the supplementary material to understand the content when I learn this  this course.

\subsection{Position and momentum operators in quantum mechanics}
In quantum mechanics \textbf{position} and \textbf{momentum} are \textbf{operators}.

\begin{definition}[Position operator]
    Position operator
    \begin{align}
        \hat{X} \psi(x)
        &= x \psi(x),
    \end{align}
    where $x$ is the eigenvalue of $\hat{X}$ acting on $\psi(x)$.
\end{definition}

$\hat{X}$ and $x$ are different things, i.e., $\hat{X}$ is an operator and $x$ is a variable.

\begin{definition}[Momentum operator]
  The momentum operator is defined as
  \begin{align}
    \hat{P} \psi(x) 
    &= -i \hbar \frac{\partial}{\partial x} \psi(x).
  \end{align}
  When we consider the three dimensional and time dependent case, we have the \textbf{orthogonal components} of a three-dimensional system as
  \begin{align}
    \hat{P}_x \psi(x,y,z,t)
    &= -i \hbar \frac{\partial}{\partial x} \psi(x,y,z,t) \\
    \hat{P}_y \psi(x,y,z,t)
    &= -i \hbar \frac{\partial}{\partial y} \psi(x,y,z,t) \\
    \hat{P}_z \psi(x,y,z,t)
    &= -i \hbar \frac{\partial}{\partial z} \psi(x,y,z,t).
  \end{align}
\end{definition}

 \begin{theorem}[De Broglie principle]
  $\psi$ can be expressed as
  \begin{align}
    \psi
    &= \exp\left[i (kx - \omega t)\right].
  \end{align}
  Then we have
  \begin{align}
    \hat{P}_x \psi
    &= \hbar k \psi,
  \end{align}
  where wave-vector $k= 2 \pi / \lambda$ and  $\lambda$ is the wavelength. Thus, the eigenvalue of momentum operator is $\hbar k$.
 \end{theorem}
 \begin{proof}
\begin{align}
  \hat{P}_x \psi
  &= -i \hbar \frac{\partial}{\partial x} \psi \\
  &= -i \hbar (ik \psi) \\
  &= \hbar k \psi.
\end{align}
 \end{proof}

 \begin{proposition}[De Broglie relation]
  \label{prop: De Broglie relation}
  We have
  \begin{align}
    \lambda
    &= \frac{h}{p},
  \end{align}
  where $\lambda$ is the wavelength, $h$ is Plank constant and $p$ is momentum.
 \end{proposition}

 \begin{corollary}
  We have 
  \begin{align}
    p
    &= \hbar k.
  \end{align}
 \end{corollary}
\begin{proof}
  Start from \ref{prop: De Broglie relation}, we have
  \begin{align}
    p
    &= \frac{h}{\lambda} \\
    &= \frac{h}{2 \pi / k} \\
    &= \frac{hk}{2 \pi} \\
    &= \hbar k.
  \end{align}
\end{proof}

\begin{property}
\begin{itemize}
  \item $\hat{S} = \hat{A} + \hat{B}$ $\Longrightarrow$ $\hat{S} \psi = \hat{A} \psi + \hat{B} \psi$ 
  \item $\hat{C} = \beta \hat{B}$ $\Longrightarrow$ $\hat{C} \psi = \beta (\hat{B} \psi)$
  \item $\hat{A} = \hat{A}^\dagger$, all operators that represent observable are Hermitian (self-adjoint)
  \item $\hat{A} \hat{B} \neq \hat{B} \hat{A}$ $\Longrightarrow$ If $\hat{M} = \hat{A} \hat{B}$, $\hat{M} \psi = \hat{A} (\hat{B} \psi)$
  \item The power of an operator: $\hat{X}^2\psi = \hat{X} (\hat{X} \psi) = \hat{X}(x \psi) = x^2 \psi$
  \item The commutator: $[\hat{A}, \hat{B}] = \hat{A} \hat{B} - \hat{B} \hat{A} \neq 0$, operators generally do not commute. Commutator is also a operator, then we have $[\hat{A}, \hat{B}]\psi = \hat{A} \hat{B} \psi - \hat{B} \hat{A} \psi$.
\end{itemize}
\end{property}

\begin{theorem}[Fundamental commutator of position and momentum]
  \label{thm: commutator of position and momentum}
  We have the fundamental commutator of position and momentum as
  \begin{align}
    [\hat{X}, \hat{P}]
    &= i \hbar.
  \end{align}
\end{theorem}
\begin{proof}
  \begin{align}
    [\hat{X}, \hat{P}] \psi
    &= \hat{X}\hat{P} \psi - \hat{P} \hat{X} \psi \\
    &= \hat{X} (-i \hbar)(i k \psi) - \hat{P} (x \psi) \\
    &= \hat{X} (\hbar k \psi) - (-i \hbar (\psi + i k x \psi)) \\
    &= x \hbar k \psi - (- i \hbar \psi + \hbar k x \psi) \\
    &= i \hbar \psi.
  \end{align}
  This leads to $[\hat{X}, \hat{P}] = i \hbar$.
\end{proof}

\begin{corollary}
  We have Heisenberg Uncertainty Principle
  \begin{align}
    \Delta x \Delta p
    &\geq \frac{\hbar}{2}.
  \end{align}
\end{corollary}
\begin{proof}
  This can be proved by \ref{thm: commutator of position and momentum}.
\end{proof}

\section{Ch 12: What are generators in classical mechanics? | Maths of Quantum Mechanics}
Lagrangian $\mathcal{L}(t,x(t), \cdot{x}(t))$ represents the classical state.

Lagrangian:
\begin{align}
  \mathcal{L} = \frac{1}{2} m \dot{x}^2 - V(x)
\end{align}
Principle of stationary action.
\begin{align}
  S[\mathcal{L}]
  &= \int_{t_i}^{t_f} dt~\mathcal{L}(t,x(t),\dot{x}(t)).
\end{align}

Calculus of variations.

Euler-Lagrange equation:
\begin{align}
  \frac{\partial \mathcal{L}}{\partial x}
  &= \frac{d}{d t} \frac{\partial \mathcal{L}}{\partial \dot{x}}.
\end{align}

\begin{theorem}
  Euler-Lagrange equation is more abstract version of Newton's second law.
\end{theorem}
\begin{proof}
  
\end{proof}

\begin{theorem}
  \begin{itemize}
    \item momentum is the generator of spatial change
    \begin{align}
      \frac{\partial \mathcal{L}}{\partial x}
      &= \frac{d}{d t} p.
    \end{align}
    \item position is the generator of momentum change
    \begin{align} 
      \frac{\partial \mathcal{L}}{\partial p}
      &= \frac{d}{d t} x.
    \end{align}
    \item energy is the generator of time change
    \begin{align}
      \frac{\partial \mathcal{L}}{\partial t}
      &= - \frac{d}{d t} E.
    \end{align}
    \item angular momentum is the generator of rotational change
    \begin{align}
      \frac{\partial \mathcal{L}}{\partial \theta}
      &= \frac{d}{d t} L.
    \end{align}
  \end{itemize}
\end{theorem}
\begin{proof}
  
\end{proof}

How is this related to quantum mechanics: recall we think about $\mathcal{L} \Longleftrightarrow \vert \psi \rangle$. They are both mathematical objects that contains all the information about the partical!

\subsection{Ch 13: Where does the Schrödinger equation come from? | Maths of Quantum Mechanics}
Time evolution in quantum mechanics?
\begin{theorem}
  The Schrödinger equation is derived as
  \begin{align}
    i \hbar \frac{d}{d t}\vert \psi \rangle
    &= \hat{H} \vert \psi \rangle.
  \end{align}
\end{theorem}
\begin{proof}
  
\end{proof}

\subsection{SPDC}
This content is from 'Q-Leap Edu Quantum Communications' from YouTube.

Spontaneous parametric down-conversion (SPDC) is a process useful in generating entangled pairs of photons.


Use linear polarization to encode a qubit.


How do polarized photons interact with a BBO? That depends on the optical axis of the crystal.


Here we have two figures to show this.


Polarization of the photon pairs is opposite to the pump!


How do we create entangled photons?


\begin{itemize}
  \item BBO crystals must be very thin to ensure indistinguishability.
  \item SPDC is a rare process. Expect 1 photon pair per $10^6$ pump photons in state-of-the-art experiments.
\end{itemize}

\subsection{Schrodinger vs. Heisenberg pictures of quantum mechanics}
This content is from 'Professor M does Science' from YouTube.

\paragraph{Schrodinger picture: state}
$t_0: \vert \psi_s(t_0) \rangle \Longrightarrow t: \vert \psi_s(t) \rangle$. From the postulates of quantum mechanics, the evolution of state is described by Schrodinger picture:
\begin{align}
    i \hbar \frac{d}{d t} \vert \psi_s(t) \rangle
    &= \hat{H}_s(t) \vert \psi_s(t) \rangle.
\end{align}

\paragraph{Schrodinger picture: observable}
\begin{itemize}
    \item Time independent observable e.g.
    \begin{align}
        &\textrm{momentum operator}~\hat{P}_s \longrightarrow - i \hbar \frac{d}{d x} \\
        &\textrm{kinetic energy operator}~\frac{\hat{P}_s}{2 m} \longrightarrow - \frac{\hbar^2}{2 m }\frac{d^2}{d x^2}.
    \end{align}
    \item Time dependent observable e.g.
    \begin{align}
        \hat{H}_s(t)
        &= \frac{\hat{P}_s^2}{2 m} + V(\hat{x}_s) \sin(\omega t).
    \end{align}
\end{itemize}

\paragraph{Time evolution operator}
We have
\begin{align}
    \vert \psi_s(t) \rangle 
    &= \hat{U}(t, t_0) \vert \psi_s(t_0) \rangle,
\end{align}
where subindex $s$ means we work in Schrödinger picture, $\hat{U}(t,t_0)$ is the time evolution operator which depends on the intial time $t_0$ and time $t$. This is the same of Schrodinger equation. They are both in Schrodinger picture. Combine the time evolution operator and Schrodinger picture, we have
\begin{align}
    \label{eq: combine the time evolution operator and Schrodinger picture}
    i \hbar \frac{d}{d t}\hat{U}(t, t_0)
    &= \hat{H}_s(t) \hat{U}(t,t_0),
\end{align}
satisfying $\hat{U}(t_0,t_0) = I$. For the time independent Hamiltonian $\hat{H}_s(t) = \hat{H}_s$, we have
\begin{align}
    \hat{U}(t,t_0)
    &= \exp\left[-i \hat{H}_s \frac{t-t_0}{\hbar} \right].
\end{align}
For time dependent case, we need Dyson series.

\paragraph{Time evolution in quantum mechanics: alternative picture}
From Schrodinger picture,
\begin{align}
    \vert \psi_s(t) \rangle 
    &= \hat{U}(t, t_0) \vert \psi_s(t_0) \rangle.
\end{align}
We then have the expectation value of observable $\hat{A}_s$ as
\begin{align}
    \langle \psi_s(t) \vert \hat{A}_s \vert \psi_s(t) \rangle
    &= \langle \psi_s(t_0) \vert \hat{U}^\dagger(t,t_0) \hat{A}_s \hat{U}(t,t_0) \vert \psi_s(t_0) \rangle \\
    &= \langle \psi_s(t_0) \vert \hat{A}_H(t) \vert \psi_s(t_0) \rangle \\
    &= \langle \psi_H(t_0) \vert \hat{A}_H(t) \vert \psi_H(t_0) \rangle,
\end{align}
where $\hat{A}_s$ is the observable in Schrödinger picture, $\vert \psi_H \rangle$ is the state in Heisenberg picture, and $\hat{A}_H$ is the observable in Heisenberg picture.

\paragraph{Unitary operator/tranaformation}
Unitary operator: $\hat{U}^{-1} = \hat{U}^\dagger$ satisfying $\hat{U}^\dagger \hat{U} = \hat{U} \hat{U}^\dagger = I$.

Unitary transformation: 
\begin{align}
    \vert \psi \rangle
    &= \vert \psi^\prime \rangle = \hat{U} \vert \psi \rangle \\
    \vert \phi \rangle
    &= \vert \phi^\prime \rangle = \hat{U} \vert \phi \rangle \\
    &\Longrightarrow \langle \psi^\prime \vert \phi^\prime \rangle = \langle \psi \vert \phi \rangle,
\end{align}
which shows unitary transformation preserves inner product.

For observable $\hat{A}$, we have
\begin{align}
    &\hat{A} \Longrightarrow \hat{A}^\prime = \hat{U} \hat{A} \hat{U}^\dagger \\
    &\Longrightarrow \langle \phi^\prime \vert \hat{A}^\prime \vert \psi^\prime \rangle = \langle \phi \vert \hat{A} \vert \psi \rangle.
\end{align}

\paragraph{Why we can have different pictures of time evolution in quantum mechanics}
Prediction of quantum mechanics: inner products e.g.
\begin{align}
    \vert \psi_s \rangle, \hat{A}_s \vert (u_n)_s \rangle = \lambda_n \vert (u_n)_s \rangle \\
    P(\lambda_n) = \vert \langle (u_n)_s \vert \psi_s \rangle \vert^2 \\
    \langle \hat{A} \rangle = \langle \psi_s \vert \hat{A}_s \vert \psi_s \rangle,
\end{align}
where $n$ denote $n^{th}$ eigenstate, this leads to the fact that unitary transformations do not change the above results. Therefore,
\begin{align}
    &\textbf{unitary transformation to Schrodinger picture} \\
    &\Longrightarrow \textbf{alternative picture of quantum mechanics}
\end{align}

\paragraph{Heisenberg picture} 
\begin{definition}
    \label{def: heisenberg picture}
    In the Heisenberg picture, quantum states are time independent.
\end{definition}

\begin{lemma}
    \label{lemma: state relation}
    The relation between the state in Schrodinger picture and the state in Heisenberg picture is 
    \begin{align}
        \vert \psi_H \rangle
        &= \hat{U}^\dagger(t,t_0) \vert \psi_s(t)\rangle \\
        &= \vert \psi_s(t_0) \rangle,
    \end{align}
    where $\vert \psi_H \rangle$ is the state in Heisenberg picture, $\vert \psi_s(t) \rangle$ is the state in the Schrodinger picture and $\vert \psi_s(t_0) \rangle$ is the initial state in the Schrödinger picture.
\end{lemma}
\begin{proof}
    Recall the time evolution:
    \begin{align}
        \vert \psi_s(t) \rangle
        &= \hat{U}(t,t_0) \vert \psi_s(t_0) \rangle.
    \end{align}
    By definition \ref{def: heisenberg picture}, we perform $\hat{U}^{-1}(t,t_0)$ on $\vert \psi_s(t) \rangle$ to get the time independent state
    \begin{align}
        \hat{U}^{-1}(t,t_0) \vert \psi_s(t) \rangle
        &= \hat{U}^{-1}(t,t_0) \hat{U}(t,t_0) \vert \psi_s(t_0) \rangle \\
        &= \vert \psi_s(t_0) \rangle.
    \end{align}
    Since $\hat{U}^{-1}(t,t_0) = \hat{U}^\dagger(t,t_0)$, we have
    \begin{align}
        \vert \psi_H \rangle
        &:= \hat{U}^\dagger(t,t_0) \vert \psi_s(t) \rangle \\
        &= \vert \psi_s(t_0) \rangle.
    \end{align}
\end{proof}

\begin{lemma}
    \label{lemma: observable relation}
    The relation of the observable in the Schrodinger and Heisenberg picture is
    \begin{align}
        \hat{A}_H(t)
        &= \hat{U}^\dagger(t,t_0) \hat{A}_s(t) \hat{U}(t,t_0),
    \end{align}
    where $\hat{A}_H$ is the observable in the Heisenberg picture and $\hat{A}_s$ is the observable in the Schrödinger picture.
\end{lemma}
\begin{proof}
    Recall the unitary transformation
    \begin{align}
        \hat{A}^\prime
        &= \hat{U} \hat{A} \hat{U}^\dagger,
    \end{align}
    and $\hat{U} = \hat{U}^\dagger(t,t_0)$. Then we define $\hat{A}_H = \hat{A}^\prime$, and we get 
    \begin{align}
        \hat{A}_H(t)
        &= \hat{U}^\dagger(t,t_0) \hat{A}_s(t) \hat{U}(t,t_0).
    \end{align}
\end{proof}

\begin{lemma}
    The expectation value of the observable in Schrodinger picture and Heisenberg picture are the same
    \begin{align}
        \langle \psi_s(t) \vert \hat{A}_s(t) \vert \psi_s(t) \rangle 
        &= \langle \psi_H \vert \hat{A}_H(t) \vert \psi_H\rangle.
    \end{align}
\end{lemma}
\begin{proof}
    \begin{align}
        &\qquad\eqnote{Insert the identy $\hat{U}(t,t_0) \hat{U}^\dagger(t,t_0) = I$} \nonumber \\
        \langle \psi_s(t) \vert \hat{A}_s(t) \vert \psi_s(t) \rangle 
        &= \langle \psi_s(t)\vert \hat{U}(t,t_0) \hat{U}^\dagger(t,t_0) \hat{A}_s(t) \hat{U}(t,t_0) \hat{U}^\dagger(t,t_0) \vert \psi_s(t) \rangle \\
        &= \langle \psi_H \vert \hat{A}_H(t) \vert \psi_H \rangle,
    \end{align}
    where in the second equation we use the definitions of \ref{lemma: state relation} and \ref{lemma: observable relation}.
\end{proof}

\paragraph{Dynamics}
\begin{lemma}
    \label{lemma: dynamics}
    We have the dynamics in Heisenberg picture as
    \begin{align}
        i \hbar \frac{d}{d t} \hat{A}_H(t) 
        &= [\hat{A}_H(t), \hat{H}_H(t)] + i \hbar \left[\frac{d}{d t} \hat{A}_s(t) \right]_H.
    \end{align}
\end{lemma}
\begin{proof}
    \begin{align}
        \frac{d}{d t} \hat{A}_H(t)
        &= \frac{d}{d t} \left[\hat{U}^\dagger(t,t_0)\hat{A}_s(t) \hat{U}(t,t_0)\right] \\
        &\qquad\eqnote{Use chain rule} \nonumber \\
        &= \left[\frac{d}{d t} \hat{U}^\dagger(t,t_0)\right] \hat{A}_s(t) \hat{U}(t,t_0) + \hat{U}^\dagger \left[\frac{d}{d t }\hat{A}_s(t)\right] \hat{U}(t,t_0) \\
        &+ \hat{U}^\dagger(t,t_0) \hat{A}_s(t)\left[\frac{d}{d t} \hat{U}(t,t_0)\right] \\
        &\qquad\eqnote{Use \ref{eq: combine the time evolution operator and Schrodinger picture}} \nonumber \\
        &= \left[-\frac{1}{i \hbar} \hat{U}^\dagger(t,t_0) \hat{H}_s(t) \right] \hat{A}_s(t) \hat{U}(t,t_0) + \hat{U}^\dagger(t,t_0) \left[\frac{d}{d t }\hat{A}_s(t)\right] \hat{U}(t,t_0) \\
        &+ \hat{U}^\dagger(t,t_0) \hat{A}_s(t)\left[\frac{1}{i \hbar} \hat{H}_s(t)\hat{U}(t,t_0)\right] \\
        &= \frac{1}{i \hbar} \left[ \hat{U}^\dagger(t,t_0) \hat{A}_s(t) \hat{H}_s(t)\hat{U}(t,t_0)-  \hat{U}^\dagger(t,t_0) \hat{H}_s(t) \hat{A}_s(t) \hat{U}(t,t_0)\right] \\
        &+ \hat{U}^\dagger(t,t_0) \left[\frac{d}{d t }\hat{A}_s(t)\right] \hat{U}(t,t_0) \\
        &\qquad\eqnote{Insert the identy $\hat{U}(t,t_0) \hat{U}^\dagger(t,t_0) = I$} \nonumber \\
        &= \frac{1}{i \hbar} [ \hat{U}^\dagger(t,t_0) \hat{A}_s(t) \hat{U}(t,t_0) \hat{U}^\dagger(t,t_0) \hat{H}_s(t)\hat{U}(t,t_0)\\
        &-  \hat{U}^\dagger(t,t_0) \hat{H}_s(t) \hat{U}(t,t_0) \hat{U}^\dagger(t,t_0) \hat{A}_s(t) \hat{U}(t,t_0)] + \left[\frac{d}{d t }\hat{A}_s(t)\right]_H \\
        &= \frac{1}{i \hbar} \left(\hat{A}_H(t) \hat{H}_H(t) - \hat{H}_H(t) \hat{A}_H(t) \right) + \left[\frac{d}{d t }\hat{A}_s(t)\right]_H \\
        &= \frac{1}{i \hbar} [\hat{A}_H(t), \hat{H}_H(t)]+ \left[\frac{d}{d t }\hat{A}_s(t)\right]_H,
    \end{align}
    which lead to
    \begin{align}
        i \hbar \frac{d}{d t} \hat{A}_H(t) 
        &= [\hat{A}_H(t), \hat{H}_H(t)] + i \hbar \left[\frac{d}{d t} \hat{A}_s(t) \right]_H.
    \end{align}
\end{proof}

\begin{corollary}
    When $\hat{A}_s(t) = \hat{A}_s$, we have
    \begin{align}
        i \hbar \frac{d}{d t} \hat{A}_H(t) 
        &= [\hat{A}_H(t), \hat{H}_H(t)].
    \end{align}
\end{corollary}

\paragraph{Conservative system}
\begin{definition}
    \label{def: conservative system}
    The conservative system have the following
    \begin{align}
        \hat{H}_s(t)
        &= \hat{H}_s.
    \end{align}
\end{definition}

\begin{lemma}
    \label{lemma: commutation relation}
    If we have $[\hat{B},\hat{C}]$, then we have
    \begin{align}
        [\hat{B}, F(\hat{C})] = 0,
    \end{align}
    where $F$ denote any function.
\end{lemma}

\begin{lemma}
    Assume we have a conservative system and assume a time independent observable $\hat{A}_s(t) = \hat{A}_s$ satisfying $[\hat{A}_s, \hat{H}_s] = 0$, then we have
    \begin{align}
        \hat{A}_H(t) = \hat{A}_s.
    \end{align}
\end{lemma}
\begin{proof}
    Using \ref{lemma: commutation relation} and 
    \begin{align}
        \hat{U}(t,t_0)
        &= \exp\left[-i \hat{H}_s (t-t_0) /\hbar\right],
    \end{align}
    we have
    \begin{align}
        [\hat{A}_s, \hat{U}(t,t_0)] = 0.
    \end{align}
    Then we have
    \begin{align}
        \hat{A}_H(t)
        &= \hat{U}^\dagger(t,t_0) \hat{A}_s \hat{U}(t,t_0) \\
        &\qquad\eqnote{Use $[\hat{A}_s, \hat{U}(t,t_0)] = 0$} \nonumber \\
        &= \hat{U}^\dagger(t,t_0) \hat{U}(t,t_0) \hat{A}_s \\
        &= \hat{A}_s.
    \end{align}
\end{proof}

\begin{corollary}
    The conservative system have the following
    \begin{align}
        \hat{H}_H
        &= \hat{H}_s = \hat{H}.
    \end{align}
\end{corollary}

\paragraph{Why we use the Heisenberg picture}
In the Heisenberg picture, it is easier to reach at Ehrenfest theorem.
\begin{theorem}
    The Ehrenfest theorem states that
    \begin{align}
        \frac{d}{d t} \langle \hat{A} \rangle
        &= \frac{1}{i \hbar} \langle [\hat{A}_s(t), \hat{H}_s(t)] \rangle + \langle \frac{d}{d t} \hat{A}_s(t) \rangle.
    \end{align}
\end{theorem}
\begin{proof}
    \begin{align}
        \frac{d}{d t} \langle \hat{A} \rangle
        &= \frac{d}{d t} \langle \psi_H \vert \hat{A}_H(t)\vert \psi_H \rangle \\
        &= \langle \psi_H \vert \frac{d}{d t} \hat{A}_H(t)\vert \psi_H \rangle \\
        &\qquad\eqnote{Use lemma \ref{lemma: dynamics}} \nonumber \\
        &= \frac{1}{i \hbar} \langle [\hat{A}_H(t), \hat{H}_H(t)] \rangle + \langle \left[\frac{d}{d t} \hat{A}_s(t)\right]_H \rangle.
    \end{align}
    We then use the fact that the expectation value is invariant under unitary transformation, which leads to the final result.
\end{proof}

\paragraph{Summary}
Here we have a nice table to summarize this subsection!

\section{Part 3. Quantum oscillator in phase space, Wigner function, and Gaussian states}
\subsection{Quantum oscillator in phase space}
\subsection{Wigner function}
\paragraph{Visualizing quantum states in phase space: Wigner function!}
\begin{itemize}
    \item $\hat{X}$ and $\hat{P}$'s eigenstates are not physical
    \item $\hat{X}$ and $\hat{P}$ do not possess common eigenstates
\end{itemize}
$\Longrightarrow$ impossible to prepare state where the quadrature take definite values! Most we can offer:
\begin{itemize}
    \item Position probability density function (PDF) $\langle x \vert \hat{\rho} \vert x \rangle = \int_{\mathbb{R}}{d p}~w_{\rho}(\overrightarrow{r})$,
    \item Momentum PDF $\langle p \vert \hat{\rho} \vert p \rangle = \int_{\mathbb{R}}{d x}~w_{\rho}(\overrightarrow{r})$,
\end{itemize}
where
\begin{align}
    \overrightarrow{r}
    &= \left(
        \begin{matrix}
            x \\
            p
        \end{matrix}
       \right),
\end{align}
and $x, p$ are eigenvalues of position and momentum. Is it possible to describe these through a combined PDF, i.e., \textbf{wigner function} $w_{\rho}(\overrightarrow{r})$ in phase space.

\begin{itemize}
    \item $\hat{\overrightarrow{R}} = \left(
        \begin{matrix}
            \hat{X} \\
            \hat{P}
        \end{matrix}
       \right)$, so $[\hat{R_m}, \hat{R}_n] = 2 i \Omega_{mn}$, where $\Omega = \left(
        \begin{matrix}
            0  & 1 \\
            -1 & 0
        \end{matrix}
       \right)$ is the sympletic form satisfying $\Omega = - \Omega^T = - \Omega^{-1}$.
    \item Displacement operator $\hat{D}(\overrightarrow{r}) = \exp\left[{\frac{i}{2} \hat{\overrightarrow{R}}^T \Omega \overrightarrow{r}}\right] = \exp\left[\frac{i}{2}(p \hat{X} - x \hat{P})\right]$
    \item Define quantum characteristic function $\chi_{\rho}(\overrightarrow{s}) = \langle \hat{D}(\overrightarrow{s})\rangle = \tr[\hat{\rho}\hat{D}(\overrightarrow{s})]$
\end{itemize}
Then, we have
\begin{definition}[Wigner function]
    The wigner function is defined as
    \begin{align}
        w_{\rho}(\overrightarrow{r})
        &= \int_{\mathbb{R}^2} \frac{d^2\overrightarrow{s}}{(4 \pi)^2} \exp\left[- \frac{i}{2} \overrightarrow{r}^T \Omega \overrightarrow{s}\right] \chi_{\rho}(\overrightarrow{s}).
    \end{align}
\end{definition}

\begin{remark}
    $\overrightarrow{x}^T \Omega \overrightarrow{y} = x_1 y_2 - x_2 y_1$

    \begin{align}
        &\int_{\mathbb{R}} \frac{d x}{2 \pi} \exp\left[i x p \right]
        = \delta(p) \\
        &\Longrightarrow   \int_{\mathbb{R}} \frac{d x}{4 \pi} \exp\left[\frac{i}{2} x p \right]= \delta(p)
    \end{align}

    \begin{align}
        &\int_{\mathbb{R}} \frac{d^2 \overrightarrow{r}}{(2 \pi)^2} \exp\left[i \overrightarrow{r}^T \overrightarrow{s} \right]
        = \delta^{(2)}(\overrightarrow{s}) \\
        &\Longrightarrow  \int_{\mathbb{R}} \frac{d^2 \overrightarrow{r}}{(4 \pi)^2} \exp\left[\frac{i}{2} \overrightarrow{r}^T \Omega \overrightarrow{s} \right]
        = \delta^{(2)}(\overrightarrow{s})
    \end{align}
\end{remark}

\begin{lemma}
    We have the right marginals:
    \begin{align}
        \int_{\mathbb{R}}d p~w_{\rho}(\overrightarrow{r}) 
        &= \langle x \vert \hat{\rho} \vert x \rangle.
    \end{align}
\end{lemma}
\begin{proof}
    \begin{align}
        \int_{\mathbb{R}}d p~w_{\rho}(\overrightarrow{r}) 
        &= \int_{\mathbb{R}} \frac{d^2 \overrightarrow{r}^\prime}{4 \pi} \exp\left[- \frac{i}{2} x p^\prime \right] \int_{\mathbb{R}} \frac{d p}{4 \pi} \exp\left[\frac{i}{2}p x^\prime\right] \chi_{\rho}(\overrightarrow{r}^\prime) \\
        &\qquad\eqnote{$\int_{\mathbb{R}} \frac{d p}{4 \pi} \exp\left[\frac{i}{2}p x^\prime\right] = \delta(x^\prime)$} \nonumber \\
        &= \int_{\mathbb{R}} \frac{d p^\prime}{4 \pi} \exp\left[- \frac{i}{2} x p^\prime\right] \tr[\hat{\rho} \hat{D}(0, p^\prime)] \\
        &\qquad\eqnote{$\hat{D}(0, p^\prime) = \exp\left[\frac{i}{2} p^\prime \hat{X} \right]$} \nonumber \\
        &= \int_{\mathbb{R}} \frac{d p^\prime}{4 \pi} \exp\left[- \frac{i}{2} x p^\prime\right] \int_{\mathbb{R}} d x^\prime \langle x^\prime \vert \hat{\rho} \exp\left[\frac{i}{2} p^\prime \hat{X} \right] \vert x^\prime \rangle \\
        &= \int_{\mathbb{R}} d x^\prime \int_{\mathbb{R}} \frac{d p^\prime}{4 \pi} \exp\left[\frac{i}{2} (x^\prime - x)p^\prime\right] \langle x^\prime \vert \hat{\rho} \vert x' \rangle \\
        &= \langle x \vert \hat{\rho} \vert x \rangle.
    \end{align}
\end{proof}

\begin{property}
    \begin{itemize}
        \item     
        \begin{align}
            \int_{\mathbb{R}}d p~w_{\rho}(\overrightarrow{r}) 
            &= \langle x \vert \hat{\rho} \vert x \rangle.
        \end{align}
        \item $w_{\rho}(\overrightarrow{r}) \in \mathbb{R}, \forall (\overrightarrow{r}, \hat{\rho})$
        \item $\int_{\mathbb{R}} d^2 \overrightarrow{r} w_{\rho}(\overrightarrow{r}) = 1$ 
        \item $\langle (\hat{X}^m \hat{P}^n)^{(s)} \rangle = \int_{\mathbb{R}^2} d^2 \overrightarrow{r} w_{\rho}(\overrightarrow{r}) x^m p^n$, where for example 
        \begin{align}
            (\hat{X}^2 \hat{P})^{(s)} = \frac{1}{3}\left(\hat{X}^2 \hat{P} + \hat{X} \hat{P} \hat{X} + \hat{P} \hat{X}^2 \right)
        \end{align}
        \item $\int_{\mathbb{R}^2} d^2 \overrightarrow{r} w_{\rho}^2(\overrightarrow{r}) \leq \frac{1}{4 \pi} \Longrightarrow \textrm{No divergence!}$ 
        \item Unique-state correspondence: $\rho = \int_{\mathbb{R}^2} = \frac{d^2 \overrightarrow{s}}{4 \pi} \hat{D}^\dagger(\overrightarrow{s}) \chi_{\rho}(\overrightarrow{s})$ 
        \item Wigner's original: $w_{\rho}(\overrightarrow{r}) = \int_{\mathbb{R}} \frac{d y}{4 \pi} \exp\left[- \frac{i}{2} p y\right] \langle x+y/2 \vert \hat{\rho} \vert x - y/2 \rangle$
        \item
        \begin{align}
            \tr\left[\hat{\rho_1} \hat{\rho_2}\right] 
            &= 4 \pi \int_{\mathbb{R}^2} w_{\rho_1}(\overrightarrow{r}) w_{\rho_2}(\overrightarrow{r}) \\
            &\qquad\eqnote{When $\hat{\rho_j} = \vert \psi_j \rangle \langle \psi_j \vert$} \nonumber \\
            &= \vert \langle \psi_1 \vert \psi_2 \rangle \vert^2 \\
            &\qquad\eqnote{Choose $\langle \psi_1 \vert \psi_2 \rangle = 0$} \nonumber \\
            &= 0,
        \end{align}
        which implies that \textbf{$w_{\rho}(\overrightarrow{r})$ is negative for some $(\rho, \overrightarrow{r})$}. $\Longrightarrow$ It is not a true PDF ! $\Longrightarrow$  quantum mechanics $\neq$ noise on classical mechanics.
    \end{itemize}
\end{property}

\begin{example}
    The wigner function of number states can be derived as
    \begin{align}
        w_{\vert n \rangle}(\overrightarrow{r}) 
        &= \frac{(-1)^n}{2 \pi} L_n(\overrightarrow{r}^2) \exp\left[- \frac{\overrightarrow{r}^2}{2}\right].
    \end{align}
    Negativity is a smoking gun for quantum phenomena in experiments! There is a nich figure in the video!
\end{example}
\begin{proof}
    It can be found in the lecture notes.
\end{proof}

\subsection{Gaussian states}
\begin{definition}
    Gaussian states are equivalent to statistics defined from $1^{st}$ and $2^{nd}$ order moments $\langle \hat{R}_j \rangle$, $\langle \hat{R}_j \hat{R}_l \rangle$. It's wigner function is
    \begin{align}
        w_{\rho}(\overrightarrow{r})
        &= \frac{1}{2 \pi \sqrt{det~V}} \exp\left[-\frac{1}{2} (\overrightarrow{r} - \overrightarrow{d})^T V^{-1} (\overrightarrow{r} - \overrightarrow{d})\right].
    \end{align}
\end{definition}

\begin{theorem}[Gaussian integral theorem]
    We have
    \begin{align}
        \int_{\mathbb{R}} d^N \overrightarrow{r} \exp\left[- \frac{1}{2} \overrightarrow{r}^T A \overrightarrow{r} + \overrightarrow{x}^T \overrightarrow{r}\right] 
        &= \sqrt{\frac{(2 \pi)^N}{det~A}} \exp\left[\frac{1}{2} \overrightarrow{x}^T A^{-1} \overrightarrow{x}\right].
    \end{align}
\end{theorem}

\begin{lemma}
    The mean vector of Gaussian states is
    \begin{align}
        d_j 
        &= \int_{\mathbb{R}^2} d^2 \overrightarrow{r} w_{\rho}(\overrightarrow{r}) r_j \\
        &= \langle \hat{R}_j \rangle \\
        &\longrightarrow \overrightarrow{d} = \langle \hat{\overrightarrow{R}} \rangle.
    \end{align}
\end{lemma}

\begin{lemma}
    The covariance matrix of Gaussian states is
    \begin{align}
        V_{jl}
        &= \int_{\mathbb{R}} d^2 \overrightarrow{r} w_{\rho}(\overrightarrow{r}) w_{\rho}(\overrightarrow{r}) \delta r_j \delta r_l \\
        &= \frac{1}{2} \langle \delta \hat{R}_j \delta \hat{R}_l + \delta \hat{R}_l \delta \hat{R}_j \rangle 
    \end{align}
    with $\delta \overrightarrow{r} = \overrightarrow{r} - \overrightarrow{d}$ and $\delta \hat{A} = \hat{A} - \langle \hat{A} \rangle$ ($V(A) = \langle \delta \hat{A}^2 \rangle$). These leads to covariance matrix 
    \begin{align}
        V
        &= \langle \delta \hat{\overrightarrow{R}} \delta \hat{\overrightarrow{R}}^T \rangle - i \Omega.
    \end{align}
\end{lemma}

\begin{theorem}[Gaussian moment theorem]
    \begin{align}
        \int_{\mathbb{R}^2} d^2 \overrightarrow{r}~w_{\rho}(\overrightarrow{r}) \delta r_{j1} \delta r_{j2} \dots \delta_{jN}
        &= 
        \left\{\begin{matrix}
            0,~for~odd~N \\
            \sum_{\left\{i_1,..., i_N\right\} \in all~(N-1)!!~pairings} V_{i1 i2} V_{i3 i4} \dots V_{i{N-1} i{N}}~for~each~N
        \end{matrix}
        \right.
    \end{align}
\end{theorem}

\begin{example}
    $N=4$: $\int_{\mathbb{R}^2} \overrightarrow{r} w_{\rho}(\overrightarrow{r}) \delta x \delta p \delta x \delta p = V_{12} V_{21} + V_{11} V_{22} + V_{12}V_{21}$.
\end{example}

\begin{example}
    \begin{align}
        \langle \hat{X}^3 \rangle
        &= \langle (\delta \hat{X} + \langle \hat{X} \rangle)^3 \rangle \\
        &= \langle \delta \hat{X}^3 \rangle + 2 \langle \hat{X} \rangle \langle \delta \hat{X}^2 \rangle + 2 \langle \hat{X} \rangle^2 \langle \delta \hat{X} \rangle + \langle \hat{X} \rangle^3.
    \end{align}
\end{example}


\end{document}