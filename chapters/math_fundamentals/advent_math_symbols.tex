\documentclass[../../note.tex]{subfiles}

\begin{document}

\chapter{Advent of Mathematical Symbols}
\begin{itemize}
    \item Kronecker delta:
    \begin{align}
        \delta_{ij} 
        &:= 
        \left\{ 
            \begin{array}{lc}
                1, & i=j      \\
                0, & i \neq j \\
            \end{array}
        \right.
    \end{align}
    \item Levi-Civita symbol: 
    \begin{align}
        \varepsilon_{ijk}
        &:= 
        \left\{
            \begin{array}{lc}
                1,  & (i,j,k)=(1,2,3)~{\rm or}~(2,3,1)~{\rm or}~(3,1,2)   \\
                -1, & (i,j,k)=(3,2,1)~{\rm or}~(2,1,3)~{\rm or}~(1,3,2) \\
                0,  & {\rm else}.
            \end{array}
        \right.
    \end{align}
    \begin{example}
        \begin{align}
            (a \times b)_i
            &= \sum_{j,k=1}^3 \varepsilon_{ijk} a_j b_k,
        \end{align}
        where $a,b$ are three dimensional vectors and $"\times"$ denotes cross product.
    \end{example}
    \item Nabla symbol:
    \begin{align}
        \nabla
        &:= 
        \left(
            \begin{array}{lc}
                \frac{\partial}{\partial x_1} \\
                \frac{\partial}{\partial x_2} \\
                \frac{\partial}{\partial x_3}
            \end{array}  
        \right).
    \end{align}
    \item Factorial: $n!:= n \cdot (n-1) \cdot (n-2) \cdots 2 \cdot 1$.
    
    Recursive definition: $0!:= 1$, $n!:= n \cdot (n-1)!,~ n \in \mathbb{N}$.
    \item  Gamma function:
    \begin{align}
        \Gamma(z)
        &:= \int_{0}^{\infty} x^{z-1} \cdot {\rm e}^{-x} {\rm d} x,~ {{\rm Re}(z)\geq 0}. 
    \end{align}
    \begin{property}
        \begin{align}
            \Gamma(n) = (n-1)!,~ n \in \mathbb{N}; ~ \Gamma(z+1) = z \cdot \Gamma(z).
        \end{align}
    \end{property}
    \item Composition: $(g \circ f)(x):= g(f(x))$.
    \item Sum symbol: $\sum_{k=1}^n a_k:= a_1 + a_2+ \cdots + a_n$.
    
    Recursive defintion: $\sum_{k=1}^0 a_k:= 0$, $\sum_{k=1}^n a_k := \left(\sum_{k=1}^{n-1} a_k \right) + a_n $.
    \item Product: $\Pi_{k=1}{n} a_k:= a_1 \cdot a_2 \cdot \cdots \cdot a_n$.
    
    Recursive defintion: $\Pi_{k=1}^0 a_k:= 1$, $\Pi_{k=1}^n a_k := \left( \Pi_{k=1}^{n-1} a_k \right) \cdot a_n $
    \item Restriction: $f \vert_{A}: A \rightarrow Y$. For $f: X \rightarrow Y$ and $A \subseteq X$, we define $f \vert_{A}(x) = f(x)$ for all $x \in A$.
    \item Pauli matrices:
    \begin{align}
        \sigma_1 = \left[
        \begin{matrix}
            0 & 1 \\
            1 & 0
        \end{matrix}
        \right].
        \sigma_2 = \left[
            \begin{matrix}
                0 & -i \\
                i & 0
            \end{matrix}
        \right].
        \sigma_1 = \left[
            \begin{matrix}
                1 & 0 \\
                0 & -1
            \end{matrix}
        \right].
    \end{align}
    \begin{property}
        We have $\sigma_k^2 = I$ and $\sigma_j \sigma_k - \sigma_k \sigma_j = 2 i \varepsilon_{jkl} \sigma_l$.
    \end{property}
    \item Set brackets: $\left\{f(x) \vert x \in A \right\}$.
    \begin{example}
        \begin{align}
            \left\{ 2x+1 \vert x \in \left\{0,1,2,3\right\} \right\} = \left\{1,3,5,7\right\}.
        \end{align}
    \end{example}
    \item Big $O$: $f(x) = O(g(x))$, $(x \rightarrow a)$, which means that $\vert f(x) \vert \leq M \cdot \vert g(x) \vert$, i.e., $\limsup_{x\rightarrow a}{\frac{f(x)}{g(x)}} < \infty$.
    \begin{align}
        x^2+x+2 = O(x^2),~ (x \rightarrow \infty) \\
        x^2+x+2 = O(x^3),~ (x \rightarrow \infty).
    \end{align}
    \item Binomial coefficient: 
    \begin{align}
        \binom{n}{k} 
        &= \frac{n \cdot (n-1) \cdots (n-k-1)}{k !} \\
        &= \frac{n!}{k! (n-k)!}.
    \end{align}
    \item Modulo: $x~{\rm mod}~n:= r \in [0,n)$ with $x = n \cdot q + r$ where $q$ is the integer.
    \begin{example}
        \begin{align}
            5~{\rm mod}~3 = 2 \\
            6~{\rm mod}~3 = 0 \\
            7.1~{\rm mod}~3 = 1.1 \\
            9.7~{\rm mod}~2.1 = 1.3.
        \end{align}
    \end{example}
    \item Beta function: 
    \begin{align}
        \beta(x, y)
        &:= \int_{0}^{1} t^{x-1} (1-t)~{\rm d}t,
    \end{align}
    where $x, y \in \mathbb{C}$, ${\rm Re}(x)>0$ and ${\rm Re}(y)>0$.
    \begin{lemma}[Identity between $\beta$ func. and $\Gamma$ func.]
        \label{lemma: identity between beta func. and Gamma func.}
        \begin{align}
            \beta(x,y)
            &= \frac{\Gamma(x)\cdot \Gamma(y)}{\Gamma(x+y)},
        \end{align}
        where $\Gamma(\cdot)$ is related to factorial and $\beta(x,y)$ is related to binomial coefficient.
    \end{lemma}
    \item Map arrows: $f:~ X \rightarrow Y$ where $X$ is the domain and $Y$ is the codomain. This map can also be denoted as elementwise-mapping as $x \longmapsto f(x)$.
    \begin{example}
        \begin{align}
            f
            &:= \mathbb{R} \rightarrow \mathbb{R} \\
            &x \longmapsto x^2.
        \end{align}
    \end{example}
    \item Little $o$: $f(x) = o(g(x))$, $(x \rightarrow a)$, which means $\lim_{x\rightarrow a} \vert \frac{f(x)}{g(x)} \vert = 0$.
    \begin{example}
        \begin{align}
            8 \cdot x^2 \neq o(x^2), ~(x\rightarrow \infty) \\
            8 \cdot x^2 \neq o(x^3), ~(x\rightarrow \infty).
        \end{align}
    \end{example}
    \item Outer product (Kronecker product for vectors):
    \begin{align}
        \left(
            \begin{matrix}
                v_1  \\
                v_2
            \end{matrix}
        \right)
        \otimes
        \left(
            \begin{matrix}
                w_1 & w_2 & w_3
            \end{matrix}
        \right)
        =
        \left(
            \begin{matrix}
                v_1 w_1 & v_1 w_2 & v_1 w_3  \\
                v_2 w_1 & v_2 w_2 & v_2 w_3
            \end{matrix}
        \right),
    \end{align} 
    i.e. matrix entries $(V \otimes W)_{ij} = v_i \cdot w_j$.
    \item Euler's phi function: $\phi: \mathbb{N} \rightarrow \mathbb{N}$ defined as
    \begin{align}
        \phi(n)
        &= {\rm count~numbers}~a \in \mathbb{N}~{\rm with}\\
        &{\rm (1)}~a \leq n \\
        &{\rm (2)}~{\rm gcd}(a,n) = 1 (\rm mutually~prime).
    \end{align}
    \begin{example}
        \begin{align}
            \phi(4)
            &= 2 \\
            \phi(5) 
            &= 4 \\
            \phi(p)
            &= p-1~for~p~prime.
        \end{align}
    \end{example}
    \item Laplace operator (Laplacian): 
    \begin{align}
        \Delta f(x)
        &= \frac{\partial^2 f}{\partial x_1^2}(x)+\frac{\partial^2 f}{\partial x_2^2}(x)+\frac{\partial^2 f}{\partial x_3^2}(x),
    \end{align}
    where $f: \mathbb{R}^3 \rightarrow \mathbb{R}$.
    \item Convolution: $(f \ast g)(x):= \int_{-\infty}^{\infty} f(\tau) \cdot g(x- \tau){\rm d}\tau$, where $f: \mathbb{R}\rightarrow \mathbb{R}$, $g: \mathbb{R}\rightarrow \mathbb{R}$ and $f \ast g: \mathbb{R}\rightarrow \mathbb{R}$.
    \item Heaviside function: 
    \begin{align}
        H(x)
        &:= 
        \left\{
            \begin{array}{lc}
                1, &~x\geq 0 \\
                0, &~x < 0
            \end{array}
        \right.
    \end{align}
    \begin{property}
        \begin{align}
            H'
            &= \delta.
        \end{align}
    \end{property}
    \item Quaternions: 
    \begin{align}
        \mathbb{H} \supseteq \mathbb{C},
    \end{align}
    where $a,b,c,d \in \mathbb{R}$, the element in $\mathbb{H}$ is $a+i\cdot b+j\cdot c+k\cdot d$ with $i^2=-1, j^2=-1,k^2=-1, ijk = -1$. $\mathbb{H}$ is not commutative in multiplication, i.e., $i\cdot j = - j \cdot i$.
    \item Infinity: $\infty$.
    \begin{example}
        In measure theory: $[0,\infty]$. We have
        \begin{align}
            a+\infty = \infty+a=\infty~for~a\in [a,\infty] \\      
        \end{align}
    \end{example}
    \item $~$ means equivalence reltation. For example, $x ~ y$ means $x$ is equivalent to $y$ for some conditions.

\end{itemize}

\end{document}