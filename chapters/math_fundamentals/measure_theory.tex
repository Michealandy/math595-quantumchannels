\documentclass[../../note.tex]{subfiles}

\begin{document}

\chapter{Measure Theory}
\section{Sigma algebra}
\begin{example}
    We define $\mathcal{P}(X)$ as the power set of set $X$. Assume that set $X = \left\{a,b\right\}$, the power set $P(X)$ would be $\left\{\emptyset, X, \{a\}, \{b\}\right\}$
\end{example}
\begin{definition}[Sigma algebra]
    $\mathcal{A} \subseteq P(X)$ is called a $\sigma-{\rm algebra}$:
    \begin{align}
        (a)~& \emptyset, X \in \mathcal{A} \\
        (b)~& A \in \mathcal{A} \Longrightarrow A^c:= X \text{\textbackslash} A \in \mathcal{A} \\
        (c)~& A_i \in \mathcal{A},~i \in \mathcal{N}\Longrightarrow \cup_{i=1}^{\infty} A_i \in \mathcal{A}.
    \end{align}
\end{definition}

\begin{definition}[Measurable sets]
    $A \in \mathcal{A}$ is called a $\mathcal{A}$-measurable set.
\end{definition}

\begin{example}
    \begin{align}
        (1)~&\mathcal{A} = \left\{\emptyset, X \right\} \\
        (2)~&\mathcal{A} = \left\{P(X)\right\}.
    \end{align}
\end{example}

\begin{lemma}
    Assume $A_i$ is $\sigma$-algebra on $X$, $i\in I$(index set). Then, we have $\cap_{i \in I} \mathcal{A}_i$ is also a $\sigma$-algebra on $X$.
\end{lemma}

\begin{definition}[Sigma algebra generated by $\mathcal{M}$]
    For $\mathcal{M} \subseteq P(X)$, there is a smallest $\sigma$-algebra that contains $\mathcal{M}$:
    \begin{align}
        \sigma(\mathcal{M})
        &:= \cap_{\mathcal{A}\supseteq \mathcal{M},~A~\sigma-algebra} \mathcal{A}.
    \end{align}
\end{definition}

\begin{example}
    We define $X = \left\{a,b,c,d\right\}$ and $\mathcal{M}=\left\{\left\{a\right\}, \left\{b\right\}\right\}$. Then we have 
    \begin{align}
        \sigma(\mathcal{M})
        &= \left\{\emptyset, X, \left\{a\right\}, \left\{b\right\},\left\{a,b\right\}, \left\{b,c,d\right\},\left\{a,c,d\right\}, \left\{c,d\right\}\right\}.
    \end{align}
\end{example}

\begin{definition}[Borel sigma algebra]
    Let $(X, \mathcal{T})$ be a topological space (Let $X$ be a metric space/Let $X$ be a subset of $\mathbb{R}^n$; We need "open sets".). We then define $\mathcal{B}(X)$ is the borel $\sigma$-algebra on $X$ as
    \begin{align}
        \mathcal{B}(X)
        &:= \sigma(\mathcal{T}),
    \end{align}
    which is the $\sigma$-algebra generated by the open sets $\mathcal{T}$.
\end{definition}

\section{What is a measure?}
\begin{definition}[Measure]
    $(X, \mathcal{A})$ is called a measurable space, where $X$ is a set and $\mathcal{A}$ is a $\sigma$-algebra on $X$. A map $\mu:~A \rightarrow [0,\infty]:= [0,\infty)+\left\{\infty\right\}$ is called a measure if it satisfies:
    \begin{align}
        (a)~& \mu(\emptyset) = 0 \\
        (b)~& \mu(\cup_{i=1}^{\infty}A_i) = \sum_{i=1}^{\infty} \mu(A_i)~with~A_i \cap A_j = \emptyset,~i \neq j~for~all~A_i \in \mathcal{A}. (\sigma-additive) 
    \end{align}
\end{definition}

\begin{definition}
    $(X,\mathcal{A},\mu)$ is called a measure space.
\end{definition}

\begin{example}
    Given $X$ and $\mathcal{A} = \mathcal{P}(X)$.
    \begin{itemize}
        \item Counting measure ($A \in \mathcal{A}$) is defined as
        \begin{align}
            \mu(A):= \left\{
                \begin{matrix}
                    \#A,&~A~\text{has finitely many elements} \\
                    \infty&~\text{else}
                \end{matrix}
            \right.
        \end{align}
        where $\# A$ means the number of elements in $A$.

        Calculation rules in $[0,\infty]$:
        \begin{align}
            x+\infty&:= \infty~for~all~x\in[0,\infty] \\
            x\cdot \infty&:= \infty~for~all~x\in(0,\infty] \\
            0\cdot \infty&:= 0~(\text{only true in most cases in measure theory!})
        \end{align}
    \item Dirac measure for $p \in X$ is defined as
    \begin{align}
        \delta_{p}(A)
        &:= \left\{
            \begin{matrix}
                1,~& p \in A \\
                0,~& else
            \end{matrix}
        \right.
    \end{align}
    \item We search a measure on $X \in \mathcal{R}^n$ satisfying:
    \begin{align}
        (1)~&\mu([0,1]^n) = 1 \\
        (2)~&\mu(x+A) = \mu(A)~for~all~x\in \mathcal{R}^n,
    \end{align}
    which is known as Lebesgue measure where the $\sigma$-algebra is not equal to power set.
\end{itemize}
\end{example}

\section{Not everything is lebesgue measurable}

\textbf{Measure problem:} search measure $\mu$ on $\mathcal{P}(\mathbb{R})$ with:
\begin{itemize}
    \item (1) $\mu([a, b]) = b - a,~b > a$,
    \item (2) $\mu(x+A) = \mu(A),~A \in \mathcal{P}(\mathbb{R}),~x \in \mathbb{R}$.
\end{itemize}
$\Longrightarrow$ $\mu$ does not exist.

\textbf{Claim:} Let $\mu$ be a measure on $\mathcal{P}(\mathbb{R})$ with $\mu((0, 1])<\infty$ and (2). $\Longrightarrow \mu = 0.$
\begin{proof}
    (a) Definitions: $I \in (0,1]$ with equivalence relation on $I$: $x ~ y \Longleftrightarrow x - y \in \mathbb{Q}$ i.e., $[x]:= \left\{x+r \vert r \in \mathbb{Q},~ x+r \in I \right\}$. Following this definition, we have a disjoint decomposition of $I$ into boxes, possibly uncontable many of them! We then pick one element $a_n$ from each box $[x_n]$ and form a set $A \in I$, i.e., $\left\{a_1, a_2, \cdots \right\} = A$. We have $A \in I$ with prperty:
    \begin{itemize}
        \item (1) For each $[x]$, there is an $a \in A$ with $a \in [x]$.
        \item (2) For all $a, b \in A:~a,b \in [x]\Longrightarrow a=b$.
    \end{itemize} 
    In uncountable case, the existence of $A \in I$ with the above property is guaranted by the axiom of choice of set theory.

    We define $A_n:= r_n + A$, where $(r_n)_{n \in \mathbb{N}}$ enumeration of $\mathbb{Q}_n(-1, 1]$. 
    
    (b) We then claim that $A_n \cap A_m = \emptyset \Longleftarrow n \neq m$. The proof is as follows: $x \in A_n \cap A_m$ $\Longrightarrow$ $x= r_n + a_n,~ a_n \in A$ and $x= r_m + a_m,~ a_m \in A$. $\Longrightarrow r_n + a_n = r_m + a_m \Longrightarrow a_n-a_m = r_n-r_m \in \mathbb{Q} \Longrightarrow a_n~a_m \Longrightarrow a_m, a_n \in [a_m] \Longrightarrow a_n = a_m \Longrightarrow r_n=r_m \Longrightarrow n=m$.

    (c) We claim that $(0,1] \subseteq \cup_{n \in \mathbb{N}} A_n \subseteq (-1, 2]$. The proof is as follows:

    Assume now: $\mu$ measure on $\mathcal{P}(\mathbb{R})$ with $\mu ((0,1])< \infty $ and (2).

    By (2): $\mu(1+A) = \mu(A)$ for all $n \in \mathbb{N}$.

    By (c): 
    we have 
    \begin{align}
        \label{measure problem (c)}
        \mu((0,1]) \leq \mu(\cup_{n \in \mathbb{N}}) \leq \mu((-1,2])
    \end{align}
    We know: $\mu((0,1]) =: C < \infty $. By using (2) and $\sigma$-additivity, we get $\mu((-1,2]) = \mu\left((-1,0] \cup (0,1] \cup (1,2] = 3 C\right)$. $\Longrightarrow_{\ref{measure problem (c)}, (b)} C \leq \sum_{n=1}^{\infty} \mu(A_n) \leq 3 C \Longrightarrow C \leq \sum_{n=1}^{\infty} \mu(A) \leq 3 C \Longrightarrow \mu(A) = 0 \Longrightarrow C=0$(henceL $\mu\left((0,1]\right)=0$) $\Longrightarrow \mu(\mathbb{R}) = \mu\left(\cup_{n \in \mathbb{Z}}(m,m+1]\right) = 0 \Longrightarrow \mu = 0$. 
\end{proof}

\section{Measurable maps}
\begin{definition}[Measurable maps]
    $(\Omega_1, \mathcal{A}_1)$ and $(\Omega_2, \mathcal{A}_2)$ are measurable spaces. $f: \Omega_1 \rightarrow \Omega_2$ is a measurable map w.r.t. $\mathcal{A}_1$ and $\mathcal{A}_2$  if $f^{-1}(A_2) \in \mathcal{A}_1$ for all $A_2 \in \mathcal{A}_2$.
\end{definition}

\begin{example}
    \begin{itemize}
        \item $(\Omega, \mathcal{A})$ and $(\mathbb{R}, \mathcal{B}(\mathbb{R}))$ are two measurable spaces. We define characteristic fucntion (aksi indicator function) as $\chi_{A}:\Omega \rightarrow \mathbb{R}$, where
        \begin{align}
            \chi_{A}(w)
            &:= \left\{
                \begin{matrix}
                    1,~& w \in A \\
                    0,~& w \notin A
                \end{matrix}
            \right.
        \end{align}
        For all measurable $A \in \mathcal{A}$, $\chi_A$ is a measurable map. We have
        \begin{align}
            \chi_A^{-1}(\emptyset) 
            &= \emptyset \in \mathcal{A},~ \chi_A^{-1}(\mathbb{R}) = \Omega \in \mathcal{A} \\
            \chi_A^{-1}(\left\{A\right\}) 
            &= A,~ \chi_A^{-1}(\left\{0\right\}) = A^c \in \mathcal{A}.
        \end{align}
        \item Composition of measurable maps. 
        \begin{lemma}
            $(\Omega_1, \mathcal{A_1}),~(\Omega_2, \mathcal{A_2}),~(\Omega_3, \mathcal{A_3})$ are measurable space. We define $\Omega_1 \stackrel{f}{\rightarrow}\Omega_2 \stackrel{g}{\rightarrow} \Omega_3$. Then $f,g$ are measurable implies $g \circ f$ is measurable.
        \end{lemma}
        \begin{proof}
            \begin{align}
                (g \circ f)^{-1}(A_3)
                &= f^{-1}(g^{-1}(A_3)) \\
                &\in \mathcal{A}_1
            \end{align}
        \end{proof}
    \end{itemize}
\end{example}

\paragraph{Important measurable maps}
\begin{lemma}
    $(\Omega,\mathcal{A})$ and $(\mathbb{R}, \mathcal{B}(\mathbb{R}))$ are measurable spaces. $f,g: \Omega \rightarrow \mathbb{R}$ are measurable maps indicates that $f+g,~f-g,~f \cdot g,~\vert f \vert$ are measurable maps.
\end{lemma}

\section{Lebesgue integral}
\begin{example}
    Define Characteristic function $\chi_A: X \rightarrow \mathbb{R},~A \in \mathcal{A}$. We define $I(A):= \mu(A)$. Surprisingly, $I(A)$ is nothing but the integral of $\chi_A$ over $A$.
\end{example}

\begin{definition}[Simple/Step/Staircsae functions,\dots]
    \label{simple functions}
    For $A_1, A_2,\dots, A_n \in \mathcal{A},$ and $~c_1,c_2, \cdots, c_n \in \mathbb{R}$. We define
    \begin{align}
        f(x)
        &:= \sum_{i=1}^n c_i \cdot \chi_{A_i}(x).
    \end{align}
    We then have $f(x)$ is measurable and the integraal of $f$ is defined as $I(f):= \sum_{i=1}^n c_i \mu(A_i)$.
\end{definition}
\begin{remark}
    The problem of the integral $I(f)$ is that it is undefined when $\mu(A_i) = \infty$. The problem can be solved by exclude $\infty$ by defintion or the following way.
\end{remark}

\begin{definition}[Lebesgue integral]
    Define $S^+:= \left\{f:X\rightarrow\mathbb{R} \vert f~simple~function,~f \geq 0\right\}$. $f \in S^+$ and choose representation $f(x) = \sum_{i=1}^n c_i \chi_{A_i}(x),~c_i \geq 0$. The lebesgue integral of $f$ w.r.t. $\mu$ is defined as
    \begin{align}
        \int_X f(x)~{\rm d}\mu(x) 
        &= \int_X f~{\rm d}\mu \\
        &= I(f) \\
        &= \sum_{i=1}^n c_i \cdot \mu(A_i) \\
        &= [0,\infty].
    \end{align}
\end{definition}

\begin{property}
    \begin{itemize}
        \item $I(\alpha f + \beta g) = \alpha I(f) + \beta I(g),~\alpha,\beta \geq 0$.
        \item $f \leq g \Longrightarrow I(f) \leq I(g)$ (monotomicity)
    \end{itemize}
\end{property}

\begin{definition}
    Define  a measurable map $f: X \rightarrow [0,\infty)$. $h = \sum_{i=1}^{n} c_i \cdot \chi_{A_i}$. The lebesgue integral of $f$ w.r.t. $\mu$ is defined as
    \begin{align}
        \int_X f~{\rm d}\mu
        &:= \sup\left\{I(h) \vert h \in S^
        +,~h \leq f \right\} \\
        & \in [0, \infty].
    \end{align}
    $f$ is called $\mu$-integrable if $\int_X f~{\rm d}\mu < \infty$.
\end{definition}

\begin{property}
    \label{property of lebesgue integral}
    Define measurable maps $f,g: X \rightarrow [0,\infty)$, we have
    \begin{itemize}
        \item 1. $f=g$ for $\mu$-almost everywhere(a.e.), which satisfies $\mu\left(\left\{x \in X \vert f(x) \neq g(x) \right\}\right)=$ $\Longrightarrow \int_X f~{\rm d}\mu = \int_X g~{\rm d}\mu$.
        \item 2. $f \leq g$ for $\mu$ a.e. $\Longrightarrow \int_X f~{\rm d}\mu \leq \int_X g~{\rm d}\mu$
        \item 3. $f=0$ for $\mu$-a.e. $\Longleftrightarrow$ $\int_X f~{\rm d}\mu = 0$.
    \end{itemize}
\end{property}
\begin{proof}[Proof of 2.: monotonicity]
    Let $h:= X \rightarrow [0,\infty)$ be a simple function, i.e.,
    \begin{align}
        h(x) 
        &= \sum_{i=1}^n c_i \chi_{A_i}(x) \\
        &= \sum_{t \in h(X)} t \cdot \chi_{\left\{x \in X \vert h(x) = t \right\}}.
    \end{align}
    Let $X = \tilde{X}^c \cup \tilde{X}$ with $\mu(\tilde{X}^c) = 0$,
    \begin{align}
        \tilde{h}(x)
        &:= \left\{
            \begin{matrix}
                h(x),~&x \in \tilde{X} \\
                a,~&x \in \tilde{X}^c
            \end{matrix}
        \right. \\
        \tilde{h}(x)
        &= \sum_{t \in h(X)} t \cdot \chi_{\left\{x \in \tilde{X} \vert h(x) = t \right\}} + a \cdot \chi_{\tilde{X}^c} \\
        I(\tilde{h})
        &= \sum_{t \in h(X)} t \cdot \mu(\left\{x \in \tilde{X} \vert h(x) = t \right\}) + a \cdot \mu(\tilde{X}^c) \\
        &= \sum_{t \in h(X)} t \left[\mu\left(\left\{x \in \tilde{X} \vert h(x) = t \right\}\right) + \mu \left(\left\{x \in \tilde{X}^c \vert h(x) = t \right\}\right)\right] \\
        &= \sum_{t \in h(X)} t \left[\mu\left(\left\{x \in \tilde{X} \vert h(x) = t \right\} \cup \left\{x \in \tilde{X}^c \vert h(x) = t \right\}\right)\right] \\
        I(h) 
        &= \sum_{t \in h(X) \mbox{\textbackslash} \left\{0\right\}} t \cdot \mu\left(\left\{x \in X \vert h(x) = t \right\}\right). 
    \end{align}
    We define
    \begin{align}
        \tilde{X}
        &:= \left\{x \in X \vert f(x) \leq g(x) \right\}, \\
        \mu(\tilde{X}^c) 
        &= 0 \\
        \int_X f~{\rm d}\mu 
        &= \sup\left\{I(h) \vert h \in S^+, h \leq f \right\} \\
        &= \sup\{I(\tilde{h})\vert \tilde{h} \in S^+, \tilde{h} \leq f ~on~\tilde{X}\} \\
        &\leq \sup\{I(\tilde{h}) \vert \tilde{h} \in S^+, h \leq g ~on~\tilde{X}\} \\
        &= \sup\{I({h}) \vert {h} \in S^+, h \leq g\} \\
        &= \int_X g~{\rm d}\mu.
    \end{align}
\end{proof}

\begin{theorem}[Monotone convergence theorem]
    \label{thm: monotone convergence theorem}
    $(X, \mathcal{A}, \mu)$ measurable spaces, $f_n: X \rightarrow [0,\infty]$, ($f: X \rightarrow [0,\infty]$) measurable for all $n \in \mathbb{N}$ with
    \begin{align}
        &f_1 \leq f_2 \leq f_3 \leq \cdots~~~\mu-\mbox{a.e.}\\  
        (\lim_{n \rightarrow \infty} \int_X f_n~{\rm d}\mu
        &= \int_X f~{\rm d}\mu~~~\mu-\mbox{a.e.}(x \in X))
    \end{align}
    This implies that
    \begin{align}
        \lim_{n \rightarrow \infty} \int_X f_n~{\rm d}\mu 
        &= \int_X \lim_{n \rightarrow} f_n~{\rm d}\mu.
        \label{mct *}
    \end{align}
\end{theorem}
\begin{proof}
    $\int_X f_1~{\rm d}\mu \leq \int_X f_2~{\rm d}\mu \leq \cdots$ and $\int_X f_n~{\rm d}\mu \leq \int_X f~{\rm d}\mu$ for $n \in \mathbb{N}$. Then we have
    \begin{align}
        \lim_{n \rightarrow \infty} \int_X f_n~{\rm d}\mu 
        &\leq \int_X f~{\rm d}\mu,
    \end{align}
    which is the first part of \ref{mct *}.

    Let $h$ be a simple function $0 \leq f \leq f$ and $\varepsilon > 0$. We define 
    \begin{align}
        X_n:= \left\{x\in X \vert f_n(x) \geq (1-\varepsilon)h(x) \right\}
    \end{align}
    with $\cup_{n=1}^{\infty} X_n = \tilde{X}$, and $\mu(\tilde{X}^c) = 0$. We have
    \begin{align}
        \int_X f_n~{\rm d}\mu 
        &\geq \int_{X_n} f_n~{\rm d}\mu \geq \int_{X_n} (1-\varepsilon)h~{\rm d}\mu \\
        \lim_{n \rightarrow \infty} \int_X f_n~{\rm d}\mu
        &\geq \lim_{n \rightarrow \infty} \int_{X_n} (1-\varepsilon)h~{\rm d}\mu \\
        &= \int_{\tilde{X}} (1-\varepsilon)h~{\rm d}\mu \\
        &= \int_X (1-\varepsilon)h~{\rm d}\mu.
    \end{align}
    This implies
    \begin{align}
        \lim_{n \rightarrow \infty} \int_X f_n~{\rm d}\mu 
        &\geq \int_X h~{\rm d}\mu,
    \end{align}
    since $\varepsilon > 0$ arbitrarily. Then we have
    \begin{align}
        \lim_{n \rightarrow \infty} \int_X f_n~{\rm d}\mu 
        &\geq \int_X f{\rm d}\mu,
    \end{align}
    since $h$ is arbitrary and $h \leq f$, which is second part of \ref{mct *}.
\end{proof}

\paragraph{Applictions}
    Given a series $(g_n)_{n \in \mathbb{N}}$, $g_n: X \rightarrow [0,\infty]$ measurable for all $n$. Then we have $\sum_{n=1}^\infty g_n: X \rightarrow [0,\infty]$ measurable and 
    \begin{align}
        \int_X \sum_{n=1}^{\infty} g_n~{\rm d}\mu
        &= \sum_{n=1}^{\infty} \int_X g_n~{\rm d}\mu,
    \end{align}
    which means the integral and sum can exchange.

\section{Fatou' lemma}
\begin{lemma}[Fatou' lemma]
    \label{lemma: fatou' lemma}
    Given $(X, \mathcal{A}, \mu)$ measurable space, $f_n: X \rightarrow [0,\infty]$ measurable for all $n \in \mathbb{N}$. Then we have
    \begin{align}
        \int_X \liminf_{n \rightarrow \infty} f_n~{\rm d}\mu
        &\leq \liminf_{n \rightarrow \infty} \int_X f_n~{\rm d}\mu.
    \end{align}
\end{lemma}
\begin{remark}
    $\liminf_{n \rightarrow \infty} f_n: X \rightarrow [0,\infty]$ is a function. This is
    \begin{align}
        g(x)
        &:= \left(\liminf_{n \rightarrow \infty} f_n \right)(x) \\
        &:= \lim_{n \rightarrow \infty} \left(\inf_{k \geq n} f_k(x)\right) \\
        &\in [0,\infty] \\
        g_n(x)
        &:= \inf_{k \geq n} f_k(x).
    \end{align}
    We have 
    \begin{align}
        g_1 \leq g_2 \leq g_3 \leq \cdots ,
    \end{align}
    which is monotonically increasing. All these functions are measurable.
\end{remark}
\begin{proof}
    \begin{align}
        \shortintertext{Since (\ref{thm: monotone convergence theorem}),}
        \int_X \lim_{n \rightarrow \infty}g_n~{\rm d}\mu 
        &= \lim_{n \rightarrow \infty} \int_X g_n~{\rm d}\mu \\
        &= \liminf_{n \rightarrow \infty} \int_X g_n~{\rm d}\mu.
    \end{align}
    We know that $g_n \leq f_n$ for all $n \in \mathbb{N}$. By (\ref{property of lebesgue integral}), we have
    \begin{align}
        \int_X g_n~{\rm d}\mu
        &\leq \int_X f_n~{\rm d}\mu,
    \end{align}
    for all $n \in \mathbb{N}$. Then we have
    \begin{align}
        \int_X \liminf_{n \rightarrow \infty} f_n~{\rm d}\mu
        &= \liminf_{n \rightarrow \infty} \int_X g_n~{\rm d}\mu \\
        &\leq \liminf_{n \rightarrow \infty} \int_X f_n~{\rm d}\mu.
    \end{align}
\end{proof}

\section{Lebesgue's dominated convergence theorem}
$(X, \mathcal{A}, \mu)$, $\mathcal{L}^1:= \left\{f: X \rightarrow \mathbb{R}~measurable \vert \int_X \vert f \vert^1~{\rm d}\mu < \infty \right\}$. For $f \in \mathcal{L}^1(\mu)$, write $f = f^+ - f^-$, where $f^+, f^- \geq 0$. Define $\int_X f~{\rm d}\mu:= \int_X f^+~{\rm d}\mu - \int_X f^-~{\rm d}\mu$.

\begin{theorem}[Lebesgue's dominated convergence theorem]
    \label{thm: lebesgue's dominated convergence theorem}
    $f_n: X \rightarrow \mathbb{R}$ measurable for all $n \in \mathbb{N}$. $f: X \rightarrow \mathbb{R}$ with $\stackrel{n \rightarrow \infty}{f(x)}$ for $x \in X$ ($\mu$-a.e.) and $\vert f_n \vert \leq g$ with $g \in \mathcal{L}^1(\mu)$ for all $n \in \mathbb{N}$, where $g$ is called integral majorant. Then: we have $f_1,f_2,\cdots \in \mathcal{L}^1(\mu)$, $f \in \mathcal{L}^1(\mu)$ and 
    \begin{align}
        \lim_{n \rightarrow \infty} \int_X f_n~{\rm d}\mu
        &= \int_X f~{\rm d}\mu.
    \end{align}
\end{theorem}
\begin{proof}
    \begin{align}
        \vert f_n \vert \leq g
        \stackrel{monotonicity}{\Longrightarrow} 
        &\int_X g~{\rm d}\mu < \infty \\
        \Longrightarrow
        & f_1, f_2, \cdots \in \mathcal{L}^1(\mu) \\
        \vert f \vert \leq g~for~\mu-\mbox{a.e.} \Longrightarrow
        & f \in \mathcal{L}^1(\mu)
    \end{align}
    We will show $\int_X \vert f_n - f \vert~{\rm d}\mu \stackrel{n \rightarrow \infty}{\Longrightarrow} 0$.
    \begin{align}
        \vert f_n -f \vert 
        &\leq \vert f_n \vert + \vert f \vert \leq 2g \\
        &\Longrightarrow h_n:= 2g - \vert f_n -f \vert \geq 0 
        \shortintertext{Hence: $h_n: X \rightarrow [0, \infty]$ measurable for all $n \in \mathbb{N}$. Then by (\ref{lemma: fatou' lemma}),} 
        &\Longrightarrow \int_X \liminf_{n \rightarrow \infty} h_n~{\rm d}\mu \leq \liminf_{n \rightarrow \infty} \int_X h_n~{\rm d}\mu \\
        &\Longrightarrow \int_X 2g~{\rm d}\mu \leq \int_X 2g~{\rm d}\mu - \limsup_{n \rightarrow \infty} \int_X \vert f_n -f \vert~{\rm d}\mu \\
        &\Longrightarrow 0 \leq \liminf_{n \rightarrow \infty} \int_X \vert f_n -f \vert~{\rm d}\mu \leq \limsup_{n \rightarrow \infty} \int_X \vert f_n -f \vert~{\rm d}\mu \leq 0 \\
        &\Longrightarrow \shortintertext{Limits exists and $\lim_{n \rightarrow \infty}\vert f_n -f \vert~{\rm d}\mu = 0$. We conclude that} \\
        &0 \leq \vert \int_X f_n~{\rm d}\mu - \int_X f~{\rm d}\mu \vert = \vert \int_X (f_n-f)~{\rm d}\mu \vert {\leq} \int_X \vert f_n - f \vert~{\rm d}\mu \stackrel{n \rightarrow \infty}{\longrightarrow} 0,
        \shortintertext{where the third inequality is due to the integral's triangle inequality.} \\
        &\Longrightarrow \lim_{n \rightarrow \infty} \int_X f_n~{\rm d}\mu = \int_X f~{\rm d}\mu.
    \end{align}
\end{proof}



\end{document}