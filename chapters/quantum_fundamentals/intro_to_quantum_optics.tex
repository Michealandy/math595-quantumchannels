\documentclass[../../note.tex]{subfiles}

\begin{document}

\chapter{Introduction to Quantum Optics by Immanuel}
\section{Introduction}
\begin{itemize}
    \item classicaal: classical atom and light
    \item semiclassical: quantized atom and classical light
    \item quantum mechanical: quantized atom and light
\end{itemize}

\paragraph{Light-Atom Interaction Hamiltonian}
\begin{itemize}
    \item classical dipole in eletric field: dipole moment $\overrightarrow{d} = q \overrightarrow{r}$, $U_I = - \overrightarrow{d} \cdot \overrightarrow{E}$. We have 
    \begin{align}
        \hat{H}_I 
        &= - \hat{d} \cdot \overrightarrow{E}(\overrightarrow{v_0}, t),
    \end{align}
    where $\hat{d} = q \hat{v}$ is the dipole operator.
    \item induced atomic dipole
\end{itemize}

\section{Light Atom Quantum Evolution}
\paragraph{Time Evolution}
We have the Schrodinger equation (both sides) as 
\begin{align}
    i \hbar \frac{\partial}{ \partial t} \vert \Psi(t) \rangle 
    &= (\hat{H}_0 + \hat{H}_I(t))\vert \Psi(t) \rangle,
\end{align}
where the general ansatz (assumption) is 
\begin{align}
    \vert \Psi(t) \rangle
    &= \sum_{n} c_n(t) e^{-i E_n t/\hbar} \vert n \rangle,
\end{align}
and 
\begin{align}
    \hat{H}_0 \vert n \rangle
    &= E_n \vert n \rangle
\end{align}
is the atomic eigenstates. Inserting $\vert \Psi(t) \rangle$ and $\hat{H}_0 \vert n \rangle$ into Schrodinger equation, we get
\begin{align}
    &i \hbar \sum_n \left\{\dot{c}_n e^{-i E_n t/\hbar} \vert n \rangle - \frac{i E_n}{\hbar} c_n e^{-i E_n t/\hbar} \vert n \rangle \right\} 
    = \sum_n \left\{c_n e^{-i E_n t/\hbar} \vert n \rangle + c_n e^{-i E_n t /\hbar} \hat{H}_I \vert n \rangle \right\} \\
    &\Longrightarrow i \hbar \sum_n \dot{c_n}e^{-i E_n t /\hbar} \vert n \rangle = \sum_n c_n e^{-i E_n t /\hbar} \hat{H}_I \vert n \rangle \\
    &\Longrightarrow i \hbar \dot{c_n}e^{-i E_k t /\hbar} \vert= \sum_n c_n(t) e^{-i E_n t/\hbar} \langle k \hat{H}_I(t)\vert n \rangle \\
    &\Longrightarrow i \hbar\dot{c}_k = \sum_n c_n(t) e^{-i E_{n,k} t / \hbar} \langle k \vert \hat{H}_I(t) \vert n \rangle,
\end{align}
where we use
\begin{align}
    \langle k \vert n \rangle 
    &= \delta_{k n}, \\
    E_{n,k}
    &= E_n - E_k, \\
    \omega_{nk}
    &= (E_n - E_k)/\hbar.
\end{align}
and $\langle k \vert \hat{H}_I(t) \vert n \rangle$ is the matrix element.

\section{Time Dependent Perturbation Theory}
Recall the time evolution:
\begin{align}
    i \hbar\dot{c}_k = \sum_n c_n(t) e^{-i \omega_{nk} t} \langle k \vert \hat{H}_I(t) \vert n \rangle,
\end{align}
and
\begin{align}
    \omega_{nk}
    &= (E_n - E_k)/\hbar.
\end{align}

Consider the Simplification (Perturbation Theory)
\begin{itemize}
    \item System only in state $\vert 1 \rangle$ at $t = 0$ $\Longrightarrow c_1 \vert 0 \rangle = 1$ (only the ground state $\vert 1 \rangle$),
    \item Perturbative treatment of interaction term: weak perturbation $\forall \vert c_k(t) \vert^2 <<1$.
\end{itemize}
We then have
\begin{align}
    i \hbar \dot{c}_k
    &= e^{i \omega_{1k}t} \langle k \vert \hat{H}_I(t) \vert 1 \rangle,
\end{align}
with $c_k(0) = 0$, we obtain:
\begin{align}
    c_k(t)
    &= \frac{1}{i \hbar} \int_0^t e^{-i \omega_{1k} t} \langle k \vert \hat{H}_I(t^\prime)\vert 1 \rangle{\rm d}t^\prime.
\end{align}
\begin{example}[Sinusoidal perturbation]
    Define
    \begin{align}
        \hat{H}(t)
        &= \hat{H}_I e^{-i \omega t}.
    \end{align}
    Given the figure in the video, we have
    \begin{align}
        c_k(T)
        &= \frac{1}{i \hbar} \int_{0}^T e^{i \Delta \omega t} \langle k \vert \hat{H}_I \vert 1 \rangle {\rm d}t \\
        &\Longrightarrow Transition~probability~P_{k1}(T) = \vert c_k(T) \vert^2 = \frac{1}{\hbar^2} \vert \langle k \vert \hat{H}_I \vert 1 \rangle \vert^2 Y(\Delta \omega, T),
    \end{align}
    with 
    \begin{align}
        Y(\Delta \omega, T)
        &= \frac{\sin^2(\Delta \omega T/2)}{(\Delta \omega /2)^2} \\
        &\sim {\rm sinc}^2 x,
    \end{align}
    where $\Delta \omega = \omega - \omega_{1 k}$ is the detwining.
\end{example}

Let's take a look at the sinc function $Y(\Delta \omega, T) = {\rm sinc}^2 x$. Transition for $\Delta \omega \leq \frac{2 \pi}{T}$, we have $\Delta \omega \cdot T\leq 2 \pi$, which implies
\begin{align}
    \Delta E \cdot T \leq h,
\end{align}
which is the time-frequency uncertainty. (The expression in the video seems wrong, so I make corrections abrove.) We have the following case
\begin{align}
    \frac{1}{2 \pi T} Y(\Delta \omega, T)
    &\stackrel{T \rightarrow \infty}{\rightarrow} \delta(\Delta \omega),
\end{align}
then we have
\begin{align}
    P_{k1}(T \rightarrow \infty)
    &= \frac{2 \pi}{\hbar^2} \vert \langle k \vert \hat{H}_I \vert i \rangle \vert^2 \delta(\Delta \omega) T.
\end{align}

\paragraph{Fermi's Golden Rule}
$\vert k \rangle$ Quasi continuum of final states. We have the transition probability
\begin{align}
    P_{k1} 
    &= \Gamma_{k1} T,
\end{align}
where
\begin{align}
    \Gamma_{k1}
    &= \frac{2 \pi}{ \hbar} \vert \langle k \vert \hat{H}_I \vert 1 \rangle \vert^2 \rho(E_k = E_1 + \hbar \omega) 
\end{align}
is called the Femi's Golden Rule,
\begin{align}
    \vert \langle k \vert \hat{H}_I \vert 1 \rangle \vert^2
\end{align}
is the coupling strength $\propto E_0^2$ and $\propto I$,
\begin{align}
    \rho(E_k = E_1 + \hbar \omega) 
\end{align}
is the density states which is number of availble final states to the system,
\begin{align}
    \Gamma_{k1}
    &\hat{=} Transition~Rate=\frac{{\rm d}P_{k1}}{{\rm d}T},
\end{align}
and density states
\begin{align}
    \rho(E)
    &= \frac{{\rm d}N}{{\rm d}E},
\end{align}
where $\Delta N$ is the number of states in an energy interval $\Delta E$ around energy $E_k$ and we let $\Delta E$ approaches 0.

\section{Two Level Atom (TLA)}
Given by the figure, in state $\vert 1 \rangle$, we have $E_1 = \hbar \omega_1$ and in state $\vert 2 \rangle$, we have $E_2 = \hbar \omega_2$ and $E_2 - E_1 = \hbar(\omega_2 - \omega_1) = \omega_{21}$. We have the Hamiltonian
\begin{align}
    \hat{H}
    &= \hat{H}_0 - \hat{d} \cdot E(t),
\end{align}
where 
\begin{align}
    E(t)
    &= \varepsilon E_0 \cos(\omega t),
\end{align}
where $\varepsilon$ is the polarization vector, $E_0$ is the field amplitude, and $\omega$ is the frequency of the light field.

\paragraph{Ansatz for Solving TLA}
We have
\begin{align}
    \vert \Psi(t) \rangle
    &= c_1(t) e^{-i \omega_1 t} \vert 1 \rangle +  c_2(t) e^{-i \omega_2 t} \vert 2 \rangle.
\end{align}

\paragraph{Time Evolution Amplitude}
We have
\begin{align}
    \dot{c_1}(t)
    &= i \frac{d^{\varepsilon}_{12} E_0}{\hbar} e^{- \omega_{21}} \cos(\omega t)c_2(t) \\
    \dot{c_2}(t)
    &= i \frac{d^{\varepsilon}_{12} E_0}{\hbar} e^{+ \omega_{21}} \cos(\omega t)c_1(t),
\end{align}
where 
\begin{align}
    d^{\varepsilon}_{12} 
    &= \langle 1 \vert \hat{d} \cdot \varepsilon \vert 2 \rangle \\
    &= \langle 1 \vert \hat{d} \vert 2 \rangle \cdot \varepsilon \\
    &= \langle 1 \vert \hat{d}_x \vert 2 \rangle \cdot \varepsilon_x +\langle 1 \vert \hat{d}_y \vert 2 \rangle \cdot \varepsilon_y + \langle 1 \vert \hat{d}_z \vert 2 \rangle \cdot \varepsilon_z.
\end{align}
is the Dipole Matrix Element, which is the atomic property and we assume it's real. We also define
\begin{align}
    \Omega_0
    &= \frac{d_{12}^{\varepsilon} E_0}{\hbar}
\end{align}
as the Rubi frequency.

\paragraph{Time Evolution}
Using Euler' form, we have
\begin{align}
    \dot{c_1}(t)
    &= i \frac{\Omega_0}{2} e^{- \omega_{21}} (e^{i \omega t} + e^{-i \omega t})c_2(t) \\
    \dot{c_2}(t)
    &= i \frac{\Omega_0}{2} e^{+ \omega_{21}} (e^{i \omega t} + e^{-i \omega t}) c_1(t)
\end{align}
by
\begin{align}
    \cos \alpha
    &= \frac{1}{2} (e^{i \alpha} + e^{-i \alpha})
\end{align}
and
\begin{align}
    e^{i \alpha}
    &= \cos \alpha + i \sin \alpha.
\end{align}

\paragraph{Rotating Wave Approximation}
We have
\begin{align}
    \dot{c_1}(t)
    &= i \frac{\Omega_0}{2}  (e^{+ i (\omega - \omega_{21}) t} + e^{-i (\omega + \omega_{21}) t})c_2(t) \\
    \dot{c_2}(t)
    &= i \frac{\Omega_0}{2}  (e^{- i (\omega - \omega_{21}) t} + e^{+i (\omega + \omega_{21}) t}) c_1(t),
\end{align}
and we ignore the sum frequency term and get
\begin{align}
    \dot{c_1}(t)
    &= i \frac{\Omega_0}{2}  e^{+ i (\omega - \omega_{21}) t} c_2(t) \\
    \dot{c_2}(t)
    &= i \frac{\Omega_0}{2}  e^{- i (\omega - \omega_{21}) t}  c_1(t),
\end{align}
which is a good approcimation for detwining $\delta = \omega - \omega_{21} \approx 0$.
We introduce
\begin{align}
    \tilde{c_1}(t)
    &= c_1(t) e^{-i \frac{\delta}{2} t} \\
    \tilde{c_2}(t)
    &= c_2(t) e^{+i \frac{\delta}{2} t}. \\
\end{align}
\paragraph{Ansatz Wavefunctions for TLA}
Whole time evolution in state amplitudes
\begin{align}
    \vert \Psi(t) \rangle 
    &= c_1^\prime(t) \vert 1 \rangle + c_2^\prime(t) \vert 2 \rangle.
\end{align}
Time evolution when field is off
\begin{align}
    \vert \Psi(t) \rangle 
    &= c_1^\prime(0) e^{-i \omega_1 t} \vert 1 \rangle + c_2^\prime(0) e^{-i \omega_2} \vert 2 \rangle.
\end{align}
However, this is boring. We chose different ansatz as
\begin{align}
    \vert \Psi(t) \rangle
    &= c_1(t) e^{-i\omega_1 t} \vert 1 \rangle + c_2(t) e^{-i\omega_2 t} \vert 2 \rangle \\
    &\Longleftrightarrow \vert \Psi(t) \rangle
    = c_1(t)  \vert 1 \rangle + c_2(t) e^{-i\omega_{21} t} \vert 2 \rangle, 
\end{align}
where $c_1(t)$ and $c_2(t)$ capture time evolution on top of eigenstate evolution! We now have
\begin{align}
    \vert \Psi(t) \rangle
    &= c_1(t)  \vert 1 \rangle + c_2(t) e^{-i\omega_{21} t} \vert 2 \rangle,
\end{align}
which is called the rotating frame of atom. We also have Rotating frame of light field as
\begin{align}
    \vert \Psi(t) \rangle
    &= \tilde{c_1}(t)  \vert 1 \rangle + \tilde{c_2}(t) e^{-i\omega t} \vert 2 \rangle,
\end{align}
where $\omega$ is the light frequency, $\tilde{c_1}$ and $\tilde{c_2}$ describe time evolution on top of fast light field oscilation.

\paragraph{Solving the TLA Dynamics}
We have the following equations:
\begin{align}
    \frac{d}{d t} \left(\begin{matrix}
        \tilde{c_1}(t) \\
        \tilde{c_2}(t)
    \end{matrix}\right) =  \frac{i}{2} \left(\begin{matrix}
        -\delta & \Omega_0 \\
        \Omega_0 & + \delta
    \end{matrix}\right) \left(\begin{matrix}
        \tilde{c_1}(t) \\
        \tilde{c_2}(t)
    \end{matrix}\right).
\end{align}
Considering the simplest case $\delta = 0$
\begin{align}
    \frac{d}{d t} \tilde{c_1}(t)
    &= \frac{i}{2} \Omega_0 \tilde{c_2}(t) \\
    \frac{d}{d t} \tilde{c_2}(t)
    &= \frac{i}{2} \Omega_0 \tilde{c_1}(t).
\end{align}
Take time dirivative of the first equation, then we have
\begin{align}
    \ddot{c_1}(t)
    &= - \frac{\Omega_0^2}{4} \tilde{c_1}(t),
\end{align}
the solutions of which are
\begin{align}
    \tilde{c_1}(t) = \cos(\Omega_0 t/2) \\
    \tilde{c_2}(t) = i \sin(\Omega_0 t/2)
\end{align}
for $\tilde{c_1}(0) = 1$ and $\tilde{c_2}(0) = 0$. Also we can obtain the excited state probability as
\begin{align}
    P_2(t)
    &= \vert c_2(t) \vert^2 \\
    &= \vert \tilde{c_2}(t) \vert^2.
\end{align}

\paragraph{Rabi Oscillations (Resonant Case)}
Nonlinear Response can be seen from the figure.

\paragraph{General Rabi Oscillations (with detuning)}
Given the figurem.
\begin{align}
    \vert \tilde{c_2}(t) \vert^2
    &= \frac{\Omega_0^2}{\Omega} \sin^2\left(\frac{1}{2}\Omega t\right) \\
    &= \frac{\Omega_0^2}{2 \Omega^2} \left\{1-\cos(\Omega t)\right\},
\end{align}
where $\Omega = \sqrt{\Omega_0^2 + \delta^2}$ is the effective Rabi frequency.

\paragraph{Interesting Special Cases}
a) Pi-Puls $\Omega_0 \tau = \pi$: swap population 
\begin{align}
    \vert 1 \rangle \rightarrow i \vert 2 \rangle \\
    \vert 2 \rangle \rightarrow i \vert 1 \rangle.
\end{align}

b) 2Pi-Puls $\Omega_0 \tau = 2 \pi$: flip the sign

c) Pi/2-Puls $\Omega_0 \tau = \pi/2$: superposition state

\section{Oscillating Dipoles}
\paragraph{Atomic Eigenstates}
\begin{align}
    \vert \Psi_{nlm}(t) \rangle 
    &= e^{-i E_{nlm} t/\hbar} \vert \Psi_{nlm}(0) \rangle, \\
    \hat{H_0} \vert \Psi_{nlm}(0) \rangle
    &= E_{nlm} \vert \Psi_{nlm} \rangle,
\end{align}
and the electron density is
\begin{align}
    \rho(r, \theta, \phi)
    &= \vert \Psi(r, \theta, \phi, t = 0)^2 \vert.
\end{align}

\paragraph{Atomic Dipole}
Calculate (Oscillating) Dipole Moment for Atomic Eigenstate. We denote 
$\vert 1 \rangle = \vert \Psi_{nlm} \rangle$. We have
\begin{align}
    d(t)
    &= \langle 1(t) \vert \hat{d} \vert 1(t) \rangle \\
    &= \langle \hat{d} \vert 1 \rangle \\
    &= -e \langle 1 \vert \hat{r} \vert 1 \rangle.
\end{align}
Then, 
\begin{align}
    -e \langle 1 \vert \hat{r} \vert 1 \rangle 
    &= -e \langle 1 \vert \hat{P} \hat{P}^{-1} \hat{r} \hat{P} \hat{P}^{-1} \vert 1 \rangle \\
    &= + e \langle 1 \vert \hat{r} \vert 1 \rangle,
\end{align}
which implies
\begin{align}
    \langle 1 \vert \hat{r} \vert 1 \rangle 
    &= 0.
\end{align}

\paragraph{Atomic Dipole - Superposition States}
Calculate (Oscillating) Dipole Moment for Atomic Superposition State 
\begin{align}
    \vert \Psi(0) \rangle
    &= \frac{1}{\sqrt{2}}(\vert 1 \rangle + i \vert 2 \rangle).
\end{align}
Evolution
\begin{align}
    \vert \Psi(t) \rangle
    &= \frac{1}{\sqrt{2}}(\vert 1 \rangle + i e^{- i \omega_{21} t} \vert 2 \rangle).
\end{align}
We have
\begin{align}
    d(t)
    &= \langle \Psi(t) \vert \hat{d} \vert \Psi(t) \rangle \\
    &= \frac{1}{2}\left\{\langle 1 \vert \hat{d} \vert 1 \rangle + \langle 2 \vert \hat{d} \vert 2 \rangle + i e^{-i\omega_{21} t} \langle 1 \vert \hat{d} \vert 2 \rangle - i e^{-i\omega_{21} t} \langle 2 \vert \hat{d} \vert 1 \rangle\right\} \\
    &= d_{12} i \frac{1}{2} \left\{e^{-i \omega_{21} t} - e^{i \omega_{21} t}\right\} \\
    &= d_{12} \sin(\omega_{21} t),
\end{align}
where $d_{12}$ is the dipole moment amplitude, $\omega_{21}$ is the natural resonance frequency.

\paragraph{Electron Density - Superposition States}
Calculate Electron Probability Density for Superposition State. The superposition state is
\begin{align}
    \Psi(r,t)
    &= \frac{1}{\sqrt{2}} \left(\Psi_1(r) + i e^{-i \omega_{21} t} \Psi_2D(t) \right).
\end{align}
The Electron Probability Density is
\begin{align}
    \rho(r, t)
    &= \vert \Psi(r, t) \vert^2 \\
    &= \Psi^\ast \Psi \\
    &= \frac{1}{2} \left\{\vert \Psi_1(r) \vert^2 + \vert \Psi_2(r) \vert^2 + 2{\rm Re}\left(i e^{-i \omega_{21} t} \Psi_1^*(r) \Psi_2(r)\right)\right\},
\end{align}
where $2{\rm Re}\left(i e^{-i \omega_{21} t} \Psi_1^*(r) \Psi_2(r)\right)$ is the interference term.

\paragraph{Examples}
This is shown by animation and figure in the video.



\section{The Bloch Sphere}
\paragraph{General Two-Level State}
\begin{itemize}
    \item General State Description
    \begin{align}
        \vert \Psi \rangle
        &= c_1^\prime \vert 1 \rangle + c_2^\prime \vert 2 \rangle \\
        &\qquad\eqnote{Up to a global phase} \\
        &= \vert c_1^\prime \vert \vert 1 \rangle + e^{i \phi} \vert c_2^\prime \vert 2 \rangle
    \end{align}
    satisfying $\vert c_1^\prime \vert^2 + \vert c_2^\prime \vert^2 = 1$.
    \item Alternative way
    \begin{align}
        \vert \Psi \rangle
        &= \cos(\theta/2) \vert 1 \rangle + e^{i \phi} \sin(\theta/2) \vert 2 \rangle,
    \end{align}
    since $\cos(\theta/2)^2 + \sin(\theta/2)^2 = 1$.
\end{itemize}

\paragraph{Geometric Description - Bloch Sphere}
We then have
\begin{align}
    \vert \Psi \rangle
    &= \cos(\theta/2) \vert 1 \rangle + e^{i \phi} \sin(\theta/2) \vert 2 \rangle
\end{align}
with $0 \leq \theta \leq \pi$ as the latitude and $0 \leq \phi \leq 2 \pi$ as the longitude. This is the Bloch Sphere representation. The definition of $\theta$ and $\theta$ and their ranges are different from my familiar coordinate system.

\paragraph{Special States on Bloch Sphere}

\paragraph{Analogy to Spin -1/2 States}
Is is shown in the figure.

\section{Density Operator and Density Matrix}
\paragraph{The Problem}
How do we describe "imperfect state preparation" in an experiment? For example, $50 \% \vert 1 \rangle$ and $50 \% \vert 2 \rangle$. We may think of 
\begin{align}
    \vert \Psi \rangle 
    &= \frac{1}{\sqrt{2}} \left(\vert 1 \rangle + \vert 2 \rangle \right). ???
\end{align}
This is $100 \% \vert \Psi \rangle$ pure state. We need stable relative phase between the two states!

\paragraph{Optical Analogy - Controlled Phase}
The double slit problem is shown in the video.

Intensity on Detection Screen:
\begin{align}
    I 
    &\propto \vert E \vert^2 = \vert E_1 + e^{i \phi} E_2 \vert^2 \\
    &= \vert E_1 \vert^2 + \vert E_2 \vert^2 + 2 {\rm Re}\left(E_1 E_2 e^{i \phi}\right).
\end{align}
As $\phi$ varies, Interference pattern "washed out"!

We need new formalism to describe mixed states!(imperfect state preparation, spontaneous emission,\dots)

\paragraph{Density Operator and Matrix}
The description of mixed states can be handled by the density operator (matrix) formalism!

\begin{itemize}
    \item Density operator (hermitian)
    \begin{align}
        \hat{\rho}
        &= \sum p_i \vert \Psi_i \rangle \langle \Psi_i \vert \\
        \hat{\rho}
        &= I \hat{\rho} I \\
        &= \sum_{i,j} \vert i \rangle \langle i \vert \hat{\rho} \vert j \rangle \langle j \vert \\
        &= \rho_{11} \vert 1 \rangle \langle  1 \vert + \rho_{12} \vert 1 \rangle \langle 2 \vert + \rho_{21} \vert 2 \rangle \langle 1 \vert + \rho_{22} \vert 2 \rangle \langle 2 \vert,
    \end{align}
    where $I = \sum_i \vert i \rangle \langle i \vert$.
    \item Density matrix
    \begin{align}
        \rho
        &= \left[\begin{matrix}
            \rho_{11} & \rho_{12} \\
            \rho_{21} & \rho_{22}
        \end{matrix}\right],
    \end{align}
    where $\rho_{11}$ and $\rho_{22}$ are the populations, $\rho_{12}$ and $\rho_{21}$ are the coherence. Since $\rho$ is hermitian, we have
    \begin{align}
        \rho_{12}
        &= \rho_{21}^\ast.
    \end{align}
\end{itemize}

\begin{example}[Example: Density Matrix of Pure State]
    We have
    \begin{align}
        \vert \Psi \rangle
        &= \vert c_1 \vert \vert 1 \rangle + e^{i \phi} \vert c_2 \vert \vert 2 \rangle.
    \end{align}
    The corresponding density operator of the \textbf{pure state} is $\hat{\rho} = \vert \Psi \rangle \langle \Psi \vert$. Then the corresponding density matrix is 
    \begin{align}
        \rho
        &= \left[\begin{matrix}
            \vert c_1 \vert^2 & \vert c_1 \vert \vert c_2 \vert e^{-i \phi} \\
            \vert c_1 \vert \vert c_2 \vert e^{i \phi} & \vert c_2 \vert^2
        \end{matrix}\right],
    \end{align}
    where $\vert c_1 \vert \vert c_2 \vert e^{-i \phi}$ and $\vert c_1 \vert \vert c_2 \vert e^{i \phi}$ are relative phase between states $\vert 1 \rangle$ and $\vert 2 \rangle$.

    specific example:
    \begin{align}
        \vert \Psi \rangle
        &= \frac{1}{\sqrt{2}} \left(\vert 1 \rangle + \vert 2 \rangle \right),
    \end{align}
    so
    \begin{align}
        \rho
        &= \left[
        \begin{matrix}
            1/2 & 1/2 \\
            1/2 & 1/2
        \end{matrix}
        \right].
    \end{align}
\end{example}

\begin{example}[Example: Fully Incoherent Mixture]
    \begin{align}
        \hat{\rho}
        &= \frac{1}{2} \vert 1 \rangle \langle 1 \vert + \frac{1}{2} \vert 2 \rangle \langle 2 \vert
    \end{align}
    with 
    \begin{align}
        \rho
        &= \left[\begin{matrix}
            1/2 & 0 \\
            0 & 1/2
        \end{matrix}\right],
    \end{align}
    where vanishingly coherence and the phase varies from $0$ to $2 \pi$. It means that we did not control phase.
\end{example}

\paragraph{Useful Facts}
\begin{itemize}
    \item Expectation values: $\langle \hat{A} \rangle = \tr(\hat{\rho} \hat{A}) = \tr(\rho A)$
    \item Time evolution (von Neumann equation)
    \begin{align}
        i \hbar \frac{\partial \hat{\rho}}{\partial t} 
        &= [\hat{H}, \hat{\rho}]
    \end{align}
    \item Pure state:  $\tr(\rho^2) = 1$
    \item Mixed states: $\tr(\rho^2) < 1$
\end{itemize}

\section{Optical Bloch Equations}
\paragraph{Time Evolution of Density Matrix}
How to calculate time evolution of density matrix?
\begin{align}
    i \hat{\hbar} \frac{\partial \hat{\rho}}{\partial t}
    &= [\hat{H}, \hat{\rho}].
\end{align}
Assume pure state
\begin{align}
    \frac{d}{d t} \rho_{11}
    &= \frac{d}{d t}(c_1 c_1^\ast) \\
    &= \dot{c}_1 c_1^\ast + c_1 \dot{c}_1^\ast \\
    &= i \frac{\Omega_0}{2} \left(e^{i \delta t} \rho_{21} - e^{-i \delta t} \rho_{12} \right) \\
    &\qquad\eqnote{Transformation to rotality frame of light} \nonumber \\
    &= i \frac{\Omega_0}{2} \left(\tilde{\rho}_{21} - \tilde{\rho}_{12}\right),
\end{align}
where
\begin{align}
    \dot{c}_1(t)
    &= i \frac{\Omega_0}{2} e^{+i \delta t} c_2(t) \\
    \dot{c}_2(t)
    &= i \frac{\Omega_0}{2} e^{-i \delta t} c_1(t) \\
    \tilde{\rho}_{12} 
    &= e^{-i \delta t} \rho_{12} \\
    \tilde{\rho}_{21} 
    &= e^{+i \delta t} \rho_{21}.
\end{align}
Other elements obtained in analogy!
\begin{align}
    \frac{d}{d t} \rho_{11}
    &= i \frac{\Omega_0}{2} \left(\tilde{\rho}_{21} - \tilde{\rho}_{12}\right) \\
    \frac{d}{d t} \rho_{22}
    &= i \frac{\Omega_0}{2} \left(\tilde{\rho}_{12} - \tilde{\rho}_{21}\right) \\
    \frac{d}{d t} \tilde{\rho}_{12}
    &= -i \delta \tilde{\rho}_{12} + i \frac{\Omega_0}{2} \left(\rho_{22} - \rho_{11}\right) \\
    \frac{d}{d t} \tilde{\rho}_{21}
    &= +i \delta \tilde{\rho}_{21} + i \frac{\Omega_0}{2} \left(\rho_{11} - \rho_{22}\right).
\end{align}
Noting that $\tilde{\rho}_{12} = \tilde{\rho}_{21}$ due to hermitian matrix, the third and the forth equations are the same. So we have
\begin{align}
    \frac{d}{d t} \rho_{11}
    &= i \frac{\Omega_0}{2} \left(\tilde{\rho}_{21} - \tilde{\rho}_{12}\right) \\
    \frac{d}{d t} \rho_{22}
    &= i \frac{\Omega_0}{2} \left(\tilde{\rho}_{12} - \tilde{\rho}_{21}\right) \\
    \frac{d}{d t} \tilde{\rho}_{12}
    &= -i \delta \tilde{\rho}_{12} + i \frac{\Omega_0}{2} \left(\rho_{22} - \rho_{11}\right).
\end{align}
\paragraph{Optical Bloch Equations with Damping}
Phenomenological damping and spontaneous emission in the figure. Combine the decay, we have
\begin{align}
    \frac{d}{d t} \rho_{11}
    &= i \frac{\Omega_0}{2} \left(\tilde{\rho}_{21} - \tilde{\rho}_{12}\right) + \gamma \rho_{22} \\
    \frac{d}{d t} \rho_{22}
    &= i \frac{\Omega_0}{2} \left(\tilde{\rho}_{12} - \tilde{\rho}_{21}\right) -\gamma \rho_{22}\\
    \frac{d}{d t} \tilde{\rho}_{12}
    &= -i \delta \tilde{\rho}_{12} + i \frac{\Omega_0}{2} \left(\rho_{22} - \rho_{11}\right) - (\gamma/2)\tilde{\rho}_{12}.
\end{align}
We now define the inversion $w = \rho_{22} - \rho_{11}$. We have Optical Bloch Equations with Damping
\begin{align}
    \frac{d}{d t} \tilde{\rho}_{21}
    &= -\left(\gamma/2 - i \delta \right) \tilde{\rho}_{21} - \frac{i \omega \Omega_0}{2} \\
    \frac{d}{d t}\omega
    &= -\gamma \left(\omega + 1\right) - i \Omega_0 \left(\tilde{\rho}_{21} - \tilde{\rho}_{12}\right)
\end{align}
in the Density Matrix Form.

\section{Optical Bloch Equations - Dynamics and Steady State}
\paragraph{Dynamical Evolution of System}
Shown in the figure in the picture.

\paragraph{Steady State Solution}
Conditions: $\frac{d}{d t} \tilde{\rho}_{21} = 0$ and $\frac{d}{d t}\omega = 0$. Then we have the solutions
\begin{align}
    \omega
    &= - \frac{1}{1 + S} \\
    \tilde{\rho}_{21}
    &= \frac{2 \Omega_0}{2\left(\gamma/2 - -\delta \right)\left(1+S\right)} \\
    S
    &= \frac{\Omega_0^2/2}{\delta^2 + \gamma^2/4} = \frac{S_0}{1+4 \delta^2/\gamma^2} \\
    S_0
    &= \frac{2 \Omega_0^2}{\gamma^2} = \frac{I}{O_{sat}},
\end{align}
where $S$ is called the saturation parameter, $S_0$ is called resonant saturation parameter.

Limiting Cases: 
\begin{itemize}
    \item $S \leq \leq 1$: $w \rightarrow -1$ where $w = \rho_{22} - \rho_{11}$. Atom is mainly in ground state.
    \item $S \geq\geq 1$: $S \rightarrow \infty$, $w \rightarrow 0$.
\end{itemize}

\begin{itemize}
    \item Excited State Population: 
    \begin{align}
        \rho_{22} \\
        &\qquad\eqnote{Combine with $\rho_{22} + \rho_{11} = 1$} \nonumber \\
        &= \frac{1}{2} (1+w) \\
        &= \frac{S}{2(1+S)} \\
        &= \frac{S_0/2}{1+S_0+4 \delta^2/\gamma^2} \\
        &\stackrel{S_0 \rightarrow \infty, \delta = 0}{\longrightarrow} \frac{1}{2}.
    \end{align}
    \item Photon Scattering Rate: $\Gamma_{ph} = \gamma \rho_{22} = \frac{\gamma}{2} \frac{S_0}{1+S_0+4 \delta^2/\gamma^2}$. $\Gamma_{ph} \rightarrow \gamma/2$ for $S_0 \rightarrow \infty$ and $\delta = 0$. We rewrite it as
    \begin{align}
        \Gamma_{ph}
        &= \left(\frac{S_0}{1+S_0}\right)\left(\frac{\gamma/2}{1+4 \delta^2 / \gamma'^2}\right) \\
        \gamma'
        &= \gamma \sqrt{1+S_0}.
    \end{align} 
\end{itemize}
It has a figure in the video. The saturation brodening is shown in the figure.

\section{Lambert-Beer Law}
\paragraph{Attenuation of Light}
It is shown in the figure.

\paragraph{Scattered Light from Slab of Atoms}
scattered light power by slab of length $d z$
\begin{align}
    d P_{sc}
    &= \Gamma_{ph} \times n A dz \times \hbar \omega,
\end{align}
where $\Gamma_{ph}$ is the single atom photon scattering rate, $\hbar \omega$ is the energy of single atom, $n A d z$ is the number of atoms. Then we have
\begin{align}
    \frac{d P_{sc}}{d z}
    &= \Gamma_{ph} \times nA \times \hbar \omega.
\end{align}

\paragraph{Scattered Light from Slab of Atoms}
Energy conservation requires
\begin{align}
    \frac{d P}{d z}
    &= - \frac{d P_{sc}}{d z} \\
    \frac{d P}{d z}
    &= \frac{d I}{d I} A.
\end{align}

Put every thing together:
\begin{align}
    \frac{d I}{d z}
    &= -\Gamma n \hbar \omega.
\end{align}
We have
\begin{align}
    \frac{d I(z)}{d z}
    &= - n \sigma I(z),
\end{align}
where $\sigma$ is the atomic scattering cross section.

\paragraph{Lambert-Beer Law (no saturation)}
We compute the solutions
\begin{align}
    I(z)
    &= I(0)e^{-n \sigma z},
\end{align}
which is the Lambert-Beer Law of Absorption.

\paragraph{Laser induced Fluorescence}
Shown in a video.

\section{Bloch Vector}
\paragraph{Density Matrix Revisited}
Density Matrix of TLA
\begin{align}
    \rho
    &= \left(\begin{matrix}
        \rho_{11} & \rho_{12} \\
        \rho_{21} & \rho_{22}
    \end{matrix}\right)
\end{align}

Density Matrix hermitian
\begin{align}
    \rho
    &= \rho^\dagger = (\rho^T)^\ast,
\end{align}
so we have
\begin{align}
    \rho
    &= \left(\begin{matrix}
        \rho_{11} & {\rm Re}\rho_{12} + i {\rm Im}\rho_{12} \\
        {\rm Re}\rho_{12} - i {\rm Im}\rho_{12} & \rho_{22}
    \end{matrix}\right).
\end{align}

Pauli matrices are
\begin{align}
    \sigma_x
    &= \left(\begin{matrix}
        0 & 1\\
        1 & 0
    \end{matrix}\right)
    \sigma_y
    = \left(\begin{matrix}
        0 & -i\\
        i & 0
    \end{matrix}\right)
    \sigma_x
    = \left(\begin{matrix}
        1 & 0\\
        0 & -1
    \end{matrix}\right).
\end{align}
The decomposition of Density matrix into Pauli matrices
\begin{align}
    \rho
    &= \frac{1}{2}\left(I + b_x \sigma_x + b_y \sigma_y + b_z \sigma_z \right),
\end{align}
where $b_x, b_y,b_y \in \mathbb{R}$.

\paragraph{Bloch Vector}
We have the density matrix in rotating frame of light
\begin{align}
    \tilde{\rho}
    &= \left(\begin{matrix}
        \rho_{11} & \tilde{\rho}_{12} \\
        \tilde{\rho}_{21} & \rho_{22}
    \end{matrix}\right),
\end{align}
where $\tilde{\rho}_{12} = \rho_{12} e^{-i \omega t}$. We use following sign convention and have
\begin{align}
    \tilde{\rho}
    &= \frac{1}{2} \left(I + u \sigma_x - v \sigma_y - w \sigma_z \right),
\end{align}
and the bloch vector is defined as
\begin{align}
    \left(\begin{matrix}
        u \\
        v \\
        w
    \end{matrix}\right).
\end{align}
It can be easily shown that
\begin{align}
    u
    &= 2 {\rm Re}(\tilde{\rho}_{12}) = \tilde{\rho}_{12} + \tilde{\rho}_{12}^\ast \\
    v
    &= 2 {\rm Im}(\tilde{\rho}_{12}) = i (\tilde{\rho}_{12}^\ast + \tilde{\rho}_{12}) \\
    w
    &= \rho_{22} - \rho_{11}, \\
\end{align}
where $u$ is the dispersive component, $v$ is the absorption component and $w$ is the inversion.

Bloch vector can be used to describe any state of TLA density matrix!

Properties of Bloch Vector
\begin{itemize}
    \item Mixed State: $u^2 + v^2 +w^2 < 1 $
    \item Pure State: $u^2 + v^2 +w^2 = 1 $
\end{itemize}

\section{Understanding Bloch Vector}
What physical behaviour do the components stand for?

\begin{itemize}
    \item $w=-1$ atom in ground state. $w = +1$ atom in excited state.
    \item What about $u,v$?
    \begin{align}
        \langle \hat{d}_i(t)
        &= \tr(\hat{\rho} \hat{d}) \\
        &= \tr\left[\left(\begin{matrix}
            \rho_{11} & \rho_{12} \\
            \rho_{12}^\ast & \rho_{22} 
        \end{matrix}\right) \left(\begin{matrix}
            0 & d_{12}^i \\
            d_{12}^i & 0
        \end{matrix}\right)\right],
    \end{align}
    where $d_{12}^x = \langle 1 \vert -q \hat{x} \vert 2 \rangle$.

    Written in the vector form, we have
    \begin{align}
        \langle \hat{d}  \rangle(t)
        &= d_{12} \left(\rho_{12} + \rho_{12}^\ast \right) \\
        &= d_{12} \left(\tilde{\rho}_{12} e^{i \omega t} + \tilde{\rho}_{12}^\ast e^{-i \omega t}\right) \\
        &= d_{12} \left[u\cos(\omega t) - v \sin(\omega t) \right],
    \end{align}
    where we use $\rho_{12} = \tilde{\rho}_{12} e^{i \omega t}$, $u$ denotes in phase and $v$ denotes $90^\circ$ out of phase component.

    Reminder: $E(t) = \epsilon E_0 \cos(\omega t)$.
    \item Which component responsible for absorption/emission? We have a figure in the video to show the classical picture.
    
    Average absorbed power per atom (classical ensemble average)
    \begin{align}
        \langle \frac{d W}{d t} 
        &= \epsilon E_0 \cos(\omega t)\langle -q \frac{d r}{d t} \rangle \\
        &= \epsilon E_0 \cos(\omega t) \langle \dot{d} \rangle.
    \end{align}

    Quantum mechanical analogue (Ehrenfest)
    \begin{align}
        \langle \frac{d W}{d t} \rangle
        &= \epsilon E_0 \cos(\omega t) \langle \dot{d} \rangle \\
        \langle \hat{d} \rangle(t)
        &= d_{12} [u \cos(\omega t) - v \sin(\omega t)]. \\
        \langle \frac{d W}{d t} \rangle
        &= -d_{12} \cdot \epsilon E_0 \omega (u \cos(\omega t) \sin(\omega t) + v \sin(\omega t)^2) \\
        \overline{\langle \frac{d W}{d t} \rangle}
        &= \frac{1}{T} \int dt \langle \frac{d W}{d t} \rangle \\
        &= - \frac{d_{12} \cdot \epsilon E_0 \omega v}{2} \\
        &= - \hbar \frac{d_{12} \epsilon E_0}{\hbar} \omega \frac{v}{2} \\
        &= - \hbar \Omega_0 \omega \frac{v}{2},
    \end{align}
    which is the absorption.
\end{itemize}

\section{Optical Bloch Equations using Bloch Vector}


\section{Interlude: The Mach-Zehnder Interferometer}

\section{Ramsey Interferometer}

%\section{Review: QM of the Harmonic Oscillator}

%\section{Wave equation and energy density of classical radiation field}

\section{Quantization of the e.m. field}
\paragraph{Fundamental Idea}
RadiationMode $(k, \alpha)$
\begin{itemize}
    \item \textbf{To every radiation mode, we associate a harmonic oscillator!} Creation and annihilation operators can change the degree of excitation of mode (occupation with photons) 
    \item \textbf{A photon is an excitation quantum of the harmonic oscillator associated with a mode!}
\end{itemize}

\paragraph{Creation and Annihilation Operators}
$\hat{a}_k \vert n_k \rangle = \sqrt{n_k} \vert n_k -1 \rangle$: decrease photon number by one photon.

$\hat{a}_k^\dagger \vert n_k \rangle = \sqrt{n_k + 1}  \vert n_k +1 \rangle$: increase photon number by one photon.

\textbf{Number operator}: $\hat{n}_k \vert n_k \rangle = n_k \vert n_k \rangle$.

\textbf{Fock state}: $\vert n_k \rangle$.


%\section{Field state of single radiation field mode: Fock States}

%\section{Field state of single radiation field mode: Coherent States}

%\section{Quadrature Operators and Phase Space of Field States}

%\section{Thermal Radiation States and Planck's Black Body Radiation Formula}

%\section{The Classical Beamsplitter}

%\section{The Quantum Beamsplitter}

%\section{The Quantum Mach-Zehnder Interferometer}

%\section{Balanced Homodyne Detection}

%\section{Quantized Light-Atom Interaction}

%\section{Jaynes-Cummings Model}

%\section{Interaction of TLA with a Coherent State}

%\section{Primer on Cabity Quantum Electrodynamics}

%\section{Seeing a Single Photon without Destroying it}

%\section{Dressed States}



\end{document}