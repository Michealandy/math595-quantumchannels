\documentclass[../../note.tex]{subfiles}

\begin{document}

\chapter{Classes of Quantum Channels}

\section{Qubit channels}
Note: please see Nielsen and Chuang for figures depicting these channels' actions on the Bloch sphere. I may make nice ones at some point but for now, I am too lazy.
\begin{enumerate}
    \item \textbf{Bit flip channel.} First, we look at the classical bit flip channel, the properties of which were established by Claude Shannon in his seminal 1948 paper. The quantum version is simply given by the Pauli $X$
    operator,
    \begin{align}
        X&= \begin{pmatrix}
        0 & 1 \\
        1 & 0
        \end{pmatrix},
    \end{align}
    which acts as
    \begin{align}
        X\ket{0} &= \ket{1},\\
        X\ket{1} &= \ket{0}.
    \end{align}
    It's eigenbasis is $\ket{\pm}:=\frac{1}{\sqrt{2}}(\ket{0}\pm\ket{1})$. So, $X\ket{+} \ket{+}$ and $X\ket{-}=-\ket{-}$. The channel acts as
    \begin{align}
        \mathcal{F}_{p}^X: \rho \mapsto (1-p)\rho + p X\rho X
    \end{align}
    So, the Kraus operators are $\sqrt{1-p} \mathbb{I}, \sqrt{p} X$
    
    \item\textbf{ Phase-flip/Z-dephasing channel} 
    \begin{align}
        Z&= \begin{pmatrix}
        1 & 0 \\
        0 & -1
        \end{pmatrix}\\
        Z \ket{0} &= \ket{0}\\
        Z \ket{1} &= -\ket{1} \\
        Z \ket{+} &= \ket{-} \\
        Z \ket{-} &= \ket{+}
    \end{align}
    The channel is given as
    \begin{align}
        \mathcal{F}_{p}^Z: \rho \mapsto (1-p) \rho + p Z\rho Z
    \end{align}
    \item \textbf{Bit-phase flip/Y-dephasing channel}
    \begin{align}
    Y&=\begin{pmatrix}
        0 & -i \\
        i & 0
        \end{pmatrix}\\
        \ket{\pm i} &=\frac{1}{\sqrt{2}}(\ket{0}\pm i\ket{1})\\
        Y \ket{i} &= \ket{i}\\
        Y \ket{-i} &= -\ket{-i} \\
        Y \ket{0} &= i\ket{1} \\
        Y \ket{1} &= -i\ket{0}
    \end{align}
    The channel is then
    \begin{align}
        \mathcal{F}_{p}^Y: \rho &\mapsto (1-p)\rho +p Y\rho Y \\
        &= (1-p)\rho +p XZ\rho ZX 
    \end{align}
    
    \item \textbf{Depolarizing channel.} Corresponds to an error model in which each Pauli error occurs with equal probability. The channel is
    \begin{align}
        \mathcal{D}_p: \rho \mapsto (1-p)\rho + \frac{p}{3}\left(X\rho X + Y\rho Y + Z\rho Z\right)
    \end{align}
    with equal probability $p/3$. The Kraus operators are then $\sqrt{1-p} \mathbb{I}, \sqrt{p/3} X, \sqrt{p/3} Y, \sqrt{p/3}Z$. Thus, the Kraus rank is 4 for this channel when $p\in (0,1)$. An alternative representation is given by 
    \begin{align}
        \rho \mapsto (1-q)\rho + q \text{tr}(\rho) \frac{\mathbb{I}}{2}
    \end{align}
    This represents replacing the input state with the maximally mixed state with ``probability'' $q$ ($q$ can be greater that 1).

What about the relation $p \leftrightarrow q$? To deduce this, we can use the identity 
\begin{align}
    \frac{1}{2}\mathbb{I}_2 = \frac{1}{4} \left(\rho + X \rho X + Y\rho Y + Z \rho Z\right) \forall \rho \in B(\mathbb{C}^2)
\end{align}
We can then derive the relationship between $p$ and $q$. We have
\begin{align}
    \frac{1}{2} \mathbb{I}_2 &= \frac{1}{4}\left(\rho + X\rho X + Y \rho Y + Z\rho Z\right)\quad \forall \rho \in B(\mathbb{C}^2)
\end{align}
\begin{align}
    (1-p) \rho +p \frac{\mathbb{I}}{2}&=(1-q)\rho + \frac{q}{4}\left(\rho + X\rho X + Y\rho Y +Z \rho Z \right)\\
    \implies \Aboxed{q&=\frac{4}{3}p}
\end{align}

\item \textbf{Generalized Pauli channel.} Let $\vec{p}=(p_0,p_1,p_2,p_3)$ be a probability distribution. The generalized Pauli channel is 
\begin{align}
    \mathcal{N}_{\vec{p}}(\rho) = p_0 \rho + p_1 X \rho X + p_2 Y \rho Y + p_3 Z \rho Z
\end{align}
where we recover the depolarizing channel by setting $p_0=1-p$ and $p_i = p/3$. Note that for any $\vec{p},$ this channel is unital.

Pauli channels are interesting from an information theoretic standpoint. Classically, they are very easily understood. Quantum mechanically, they very much are not understood (except in the case of flip or dephasing channels).
\end{enumerate}

In what sense are these flip channels de-phasing? Let us look at the example of the phase flip channel
\begin{align}
    \mathcal{F}_{p}^{Z}: \rho \mapsto (1-p)\rho + p Z\rho \rho Z.
\end{align}
We can understand the justification of the term de-phasing by looking at the action on the density matrix
\begin{align}
    \begin{pmatrix}
    \rho_{11} & \rho_{12}\\
    \rho_{21} & \rho_{22}
    \end{pmatrix} \mapsto 
    \begin{pmatrix}
    \rho_{11} & (1-2p)\rho_{12}\\
    (1-2p)\rho_{21} & \rho_{22}
    \end{pmatrix}
\end{align}
where $\rho \geq 0, \text{tr}(\rho)=1 \implies \rho_{11}+\rho_{22}=1$. We can make the following observations
\begin{enumerate}
     \item If $\rho = x \ket{0}\bra{0} + (1-x)\ket{1}\bra{1}$, then 
     \begin{align}
         \mathcal{F}_p^Z(\rho)=\rho \quad \forall p\in [0,1]
     \end{align}
     \item $p=\frac{1}{2}$: the channel is diagonal in the z-basis for all input states
\end{enumerate}
We can also think about sending classical information through this channel. Let us encode $0$ as $\ket{0}\bra{0}$ and $1$ as $\ket{1}\bra{1}$. That is, we are encoding one bit in one qubit. We can do this reliably with this channel because
\begin{align}
    \mathcal{F}_p^Z (\ket{0}\bra{0}) &= \ket{0}\bra{0}, \\
     \mathcal{F}_p^Z (\ket{1}\bra{1}) &= \ket{1}\bra{1}.
\end{align}
This means that we can send one bit through the channel. The key idea is: this channel preserves classical information because it preserves the orthogonality of the basis states. The same would hold for the $X$ channel but you would have to encode info in the $\ket{\pm}$ basis. 

\begin{enumerate}
\setcounter{enumi}{5}
    \item \textbf{Amplitude damping channel.} Physical model: 2-level system (e.g. an atom with a ground state $\ket{0}$ and excited state $\ket{1}$. If the system is in an excited state $\ket{1}$, it decays with a certain probability, $\gamma$, emitting a photon to the environment. How can we capture this decay process mathematically? Consider the isometry 
    \begin{align}
        \ket{0}_A &\mapsto \ket{0}_B \ket{0}_E,\\
        \ket{1} &\mapsto \sqrt{1-\gamma} \ket{1}_B \ket{0}_E +\sqrt{\gamma}\ket{0}_B\ket{1}_E.
    \end{align}
Compactly we can write

The Kraus operators for this channel are 
\begin{align}
    K_0 &=\bra{0}_E V = \ket{0}\bra{0}_B + \sqrt{1-\gamma} \ket{1}\bra{1})_B = \begin{pmatrix}
    1 & 0 \\
    0 & \sqrt{1-\gamma}
    \end{pmatrix},\\
     K_1&= \bra{1}_E V = \sqrt{\gamma}\ket{0}_B\bra{1}_A = \begin{pmatrix}
    0 & \gamma \\
    0 & 0
    \end{pmatrix}.
\end{align}
In the Kraus representation, then, we can write the amplitude damping channel 
\begin{align}
    \mathcal{A}_{\gamma}: \rho \mapsto K_0 \rho K_0^{\dagger} + K_1 \rho K_1^{\dagger}.
\end{align}
We note that the amplitude damping channel is not unital. Therefore, it is not a Pauli (or even a mixed-unitary) channel. Interestingly, it is completely understood how to send \textit{quantum information} through the amplitude damping channel; however, it is not known how much classical information can be sent through this channel.
\item \textbf{Erasure channel.} Classical model:

Quantum version: $\mathcal{E}_p: B(\mathcal{H}) \rightarrow B(\mathcal{H} \otimes \mathbb{C})$. The channel acts as
\begin{align}
    \rho \mapsto (1-p) \rho + p \text{tr}(\rho) \ket{e}\bra{e}.
\end{align}
This channel has Kraus operators given by $K_0 \sqrt{1-p} \mathbb{I}, K_1 \sqrt{p}\ket{0}\bra{0}, K_2 \sqrt{p} \ket{e}\bra{1}.$ Note that $\ket{e}$ must be orthogonal to all $\rho \in B(\mathcal{H})$. That is, $\braket{\ket{e}\bra{e},\rho}=0 \quad \forall \rho \in B(\mathcal{H})$. Bob can always tell whether erasure happened by performing a measurement! This is a very well-understood quantum channel.
\end{enumerate}

\section{Generalized dephasing channels}
A generalized dephasing channel leaves a \textit{fixed} orthonormal basis invariant and it dephases off-diagonal elements with respect to that fixed basis. That is, we lose quantum coherences under the action of a generalized dephasing channel. 

Construction: $\mathcal{H}=\mathbb{C}^d$ with orthonormal basis $\{\ket{i}\}_{i=1}^d$. Choose an environment $\mathcal{H}_E$ with dim$\mathcal{H}_E :=|E| \geq 2$, and let $\{\varphi_i\}_{i=1}^{|E|}$ be \textit{some} set of pure states on $E$. This set is normalized but not orthogonal. Then, define the isometry
\begin{align}
   V: \ket{i}_A \mapsto \ket{i}_B \otimes \ket{\varphi_i}_E.
\end{align}
The Stinespring extension is
\begin{align}
    \mathcal{N}(\rho_A) &= \text{tr}_E V \rho_A V^{\dagger}, \\
    &= \sum_{i,j} \ket{i}\rho \bra{i} \ket{i}\bra{j}_B \text{tr}\left(\ket{\varphi_i}\bra{\varphi_j}\right),\\
    &=\sum_{i,j} \braket{\varphi_i|\varphi_j} \braket{i|\rho|j} \ket{i}\bra{i}_B.
\end{align}
Let us confirm this channel acts as expect. 
\begin{align}
    [\mathcal{N}(\rho)]_{kk} &= \braket{\varphi_k|\varphi_k} \bra{k}\rho \ket{k} = \rho_{kk}, \\
    [\mathcal{N}(\rho)]_{jk} &=\braket{\varphi_j|\varphi_k} \bra{j}\rho \ket{k},
\end{align}
but $0 < \braket{\varphi_j|\varphi_k} \leq 1$. Thus, we see that the diagonals are preserved but the off-diagonals are dephased. 

Higher-dimensional example: Let $d\geq 2$, and define two unitaries 
\begin{align}
    X\ket{i} &= \ket{i+1 \text{ mod } d} \quad \text{ (shift operator) } \\
    Z\ket{j} &= \omega^j \ket{j}, \text{ where } \quad \omega=\exp{\left(\frac{2 \pi i}{d}\right)} \quad \text{ (clock operator) }
\end{align}
This generalizes the Pauli operators: $X^d=Z^d = \mathbb{I}$. These are the generators of the Heisenberg-Weyl group:
\begin{align}
    \{\omega^j Z^k X^l : j,k,l \in [d]\}.
\end{align}
So, for $\rho \in B(\mathbb{C}^d)$ we maps like 
\begin{align}
    \rho &\mapsto (1-p)\rho + p X\rho X^{\dagger}, \\
    \rho &\mapsto (1-p)\rho +\frac{p}{3} Z\rho Z^{\dagger} + \frac{2p}{3} Z^2 \rho (Z^{\dagger})^2.
\end{align}
The generalized de-phasing channels have full classical capacity 
\begin{align}
    C(\mathcal{N})=\log{d} \quad (\text{ in general, } C(\mathcal{N})\leq \log{d})
\end{align}

Sketch of a proof: we know that for a generalized dephasing channel, there exists an orthonormal basis $\{\ket{i}\}_i$ such that 
\begin{align}
    \mathcal{N}(\ket{i}\bra{i}) = \ket{i}\bra{i}
\end{align}
of classical signals/messages $x_i, \dots, x_d$. We can use the encoding
\begin{align}
    x_i \mapsto \ket{i}\bra{i}
\end{align}
where $\braket{i|j}=\delta_{ij}$. Thus, $d$ messages can be sent perfectly. So $\log{d}$ bits of classical info can be sent through $N$. Bob can measure the output to retrieve the classical message. 

\subsection{Detour: Holevo information ($\mathcal{X}$-quantity)}
Recall: the von Neumann entropy is defined as 
\begin{align}
    S(\rho) = - \text{tr} \rho \log \rho.
\end{align}
When one has the spectral decomposition of $\rho=\sum_i \lambda_i \ket{\psi_i}\bra{\psi_i}$, the von Neumann entropy is just the Shannon entropy for the eigenvalue distribution
\begin{align}
    S(\rho) = -\sum_i \lambda_i \log \lambda_i.
\end{align}
If we then have an ensemble of quantum states $E=\{p_x, \rho_x\}$, we have
\begin{align}
    \mathcal{X}(E,\mathcal{N})=S(\sum_x p_x \mathcal{N}(\rho_x)) - \sum_x p_x S(\mathcal{N}(\rho_x))
\end{align}

The Holevo information is then defined as
\begin{align}
 \Aboxed{\text{Holevo Information, } \mathcal{X}(\mathcal{N})=\max_{E} \mathcal{X}(E,\mathcal{N})}
\end{align}
A fundamental result due to Holevo, Schumacher, and Westmoreland is 
\begin{align}
    \Aboxed{C(\mathcal{N})\geq \mathcal{X}(\mathcal{N})}
\end{align}

If $\mathcal{N}$ is a generalized dephasing channel, there exists an orthonormal basis $\{\ket{i}\}$ such that $\mathcal{N}(\ket{i}\bra{i})=\ket{i}\bra{i}$ for all $i$. Let our ensemble be 
\begin{align}
    \rho_i &= \ket{i}\bra{i} \\
    p_i &= \frac{1}{d}, 
\end{align}
which implies
\begin{align}
    \bar{\rho} = \sum_i p_i \rho_i = \frac{1}{d} \mathbb{I}
\end{align}
Then, $E_n = \{\frac{1}{d}, \rho_i \}$, so the Holevo quantity is 
\begin{align}
    \mathcal{X} (E_n, \mathcal{N}) &= S(\sum_i p_i \mathcal{N}(\rho_i))-\sum_i p_i S(\mathcal{N}(\rho_i)) \\
    &= S(\sum_i p_i \rho_i) - \sum_i p_i S(\rho_i)\\
    &=\log{d}-0 \\
    &= \log{d}
\end{align}
Thus, we have the following chain of inequalities 
\begin{align}
    \log{d} \leq \mathcal{X}(\mathcal{N}) \leq C(\mathcal{N}) \leq \log{d} \implies \Aboxed{C(\mathcal{N}) = \log{d}}
\end{align}
as desired.

\subsection{Some comments on last lecture}
\begin{itemize}
    \item The rank of the Kraus operators is not unitarily invariant. We can see this by examining the 50-50 dephasing channel. 
    \begin{align}
        \rho \mapsto \frac{1}{2} \rho + \frac{1}{2} Z \rho Z \implies K_0 = \frac{1}{\sqrt{2}} \mathbb{I},\quad K_1 = \frac{1}{\sqrt{2}} Z
    \end{align}
    We can apply a Hadamard transform to the operators to obtain new Kraus operators
    \begin{align}
    L_i &= \sum_j H_{ij}K_j, \\
       \implies L_0 &= \ket{0}\bra{0}, L_1 = \ket{1}\bra{1},
    \end{align}
    which are rank-1 operators now when the previous Kraus operators were full rank. 
    \item Every unital qubit channel is unitarily equivalent to a Pauli channel. That is, if $\mathcal{N}: B(\mathbb{C}^2)\rightarrow B(\mathbb{C}^2)$ is a unital channel, then there are unitaries $U,V$ such that 
    \begin{align}
        M(\rho)= U\mathcal{N}(V \rho V^{\dagger})U^{\dagger}
    \end{align}
    is Pauli.
    \item mixed unitary channels 
    \begin{align}
        \rho &\mapsto \sum_i p_i U_i \rho U_i^{\dagger}, \quad U_i \text{ unitary}\\
        \mathbb{I} &\mapsto \sum_i p_i U_i U_i^{\dagger} = \mathbb{I} \quad \text{ unital! }
    \end{align}
    Every unital qubit channel is mixed unitary. Watrous' book provides a nice example of a unital channel that is \textit{not} mixed unitary:
    \begin{align}
        X \in B(\mathbb{C}^3), \quad \rho \mapsto \frac{1}{2}\text{tr}(X) \mathbb{I} -\frac{1}{2} X^T
    \end{align}
\end{itemize}

\section{Entanglement-breaking channels}
Reminder: A bipartite $\rho_{AB}$ is called separable if it lies in the convex hull of product states. That is,
\begin{align}
    \rho_{AB} \in \text{conv}\{\omega_A \otimes \sigma_B : \omega_{A(B)} \text{ state on } \mathcal{H}_{A(B)}\}
\end{align}
Explicitly, 
\begin{align}
    \rho_{AB} = \sum_i p_i \omega^i_A \otimes \sigma_B^i
\end{align}

\begin{definition}
A channel $\mathcal{N}: A\rightarrow B$ is entanglement-breaking if 
\begin{align}
    (\mathbb{I}_R \otimes \mathcal{N})(\rho_{RA}) 
\end{align}
is separable for any $\rho_{RA}$.
\end{definition}

\begin{proposition}
The following are all equivalent
\begin{itemize}
    \item $\mathcal{N}: A\rightarrow B$ is entanglement breaking
    \item $\tau_{AB}^{\mathcal{N}}=(\mathbb{I}_A \otimes \mathcal{N})(\gamma_{AA'})$ is separable 
    \item $\mathcal{N}$ has a Kraus representation with rank-1 Kraus operators 
    \item $\mathcal{N}$ is a measure-and-prepare channel: there exists POVM $E=\{E_i\}_i$ and states $\{\sigma_i\}_i$ such that 
    \begin{align}
        \mathcal{N}(\rho) = \sum_i \text{tr}(\rho E_i) \sigma_i
    \end{align}
    where we recall that a POVM must satisfy $E_i \geq 0, \sum_i E_i = \mathbb{I}$.
\end{itemize}
\end{proposition}

\begin{proof} 
(a $\implies$ b): ($\mathbb{I}_R\otimes \mathcal{N}$)($\rho_{RA}$) is separable for all $\rho_{RA}$. In particular, for $\gamma_{RA}, \quad (\ket{\gamma}_{RA})=\sum_i \ket{i}_R \ket{i}_A)$.

(b$\implies$ c): $\tau_{AB}^{\mathcal{N}}$ is separable: there exist pure states $\psi_i = \ket{\psi_i}\bra{\psi_i}_A$ and $\varphi_i = \ket{\varphi_i}\bra{\varphi_i}_B$ such that $\frac{1}{d} \tau^{\mathcal{N}}_{AB}=\sum_i p_i \psi_i \otimes \varphi_i$, where $d=|A|$. Then, set $K_i = \sqrt{p_i d}\ket{\varphi_i}_B \bra{\bar{\psi}_i}_A$. We can then check the action of these operators
\begin{align}
    \frac{1}{d} \sum_i (\mathbb{I}\otimes K_i) (\gamma_{AA'})(\mathbb{I}_A \otimes K_i)^{\dagger}&= \frac{1}{d} \sum_{i,j,k} \ket{j}\bra{k}_A \otimes K_i \ket{j}\bra{k}_{A'} K_i^{\dagger} \\
    &= \sum_{i,j,k} d p_i \ket{j}\bra{k}_A \otimes \bra{\bar{\psi}_i}\ket{j}\bra{k}\ket{\bar{\psi}_i} \ket{\varphi_i}\bra{\varphi_i}_B\\
    &=\sum_{i,j,k} p_i \ket{j}\bra{k}_A \otimes \bra{\bar{\psi}_i}\ket{j}\bra{k}\ket{\bar{\psi}_i} \ket{\varphi_i}\bra{\varphi_i}_B\\
    &=\sum_i p_i \left(\sum_{j,k} \braket{k | \bar{\psi}_i | j} \ket{j} \bra{k}\right)\otimes \ket{\varphi_i}\bra{\varphi_i}_B\\
    &=\sum_i p_i (\psi_i) \otimes \varphi_i \\
    &= \frac{1}{d} \tau_{AB}^{\mathcal{N}}
\end{align}
\begin{align}
    \sum_{i} K_i^{\dagger} K_i &=\mathbb{I} \\
   &= d \sum_i p_i \ket{\bar{\psi}_i} \braket{\varphi_i | \varphi_i}\bra{\bar{\psi}_i}\\
    &= d \sum_i p_i \bar{\psi}_i \\
    &= d(\sum_i p_i \bar{\psi_i})\\
    &=d \frac{1}{d} \bar{\mathbb{I}}_A\\
    &= \mathbb{I}_A
\end{align}

(c $\implies$ d): rank($K_i$)=1: $K_i = \ket{\mathcal{X}_i}_B \bra{\omega_i}_A$ for some vectors $\ket{\mathcal{X}_i}_B ,\ket{\omega_i}_A, \braket{\mathcal{X}_i |\mathcal{X}_i}=1$.

\begin{align}
    \mathcal{N} (\rho) &= \sum_i K_i \rho K_i^{\dagger} \\
    &= \sum_i \braket{\omega_i | \rho | \omega_i} \ket{\mathcal{X}_i} \bra{\mathcal{X}_i}_B 
\end{align}
POVM: $\omega = \{\omega_i\}$, states: $\mathcal{X}_i$
\begin{align}
    \mathbb{I} &= \sum_i K_i^{\dagger}K_i \\
    &= \sum_i \ket{\omega_i} \braket{\mathcal{X}_i | \mathcal{X}_i}\bra{\omega_i} \\
    &= \sum_i \ket{\omega_i}\bra{\omega_i}
\end{align}
 
 (d $\implies$ a): Let $\rho_{RA}$ be arbitrary: 
 \begin{align}
     (\mathbb{I}_R \otimes \mathcal{N})(\rho_{RA}) &= \sum_i \text{tr}_B \left[(\mathbb{I}_A \otimes E_i)\rho_{RA}\right]\otimes \sigma_i \\
     &= \sum_i \text{tr}_B \left[(\mathbb{I}_R \otimes \sqrt{E_i}) \rho_{RA}(\mathbb{I}_R \otimes \sqrt{E_i})\right] \otimes \sigma_i \\
     &= \sum_i p_i \omega_i \otimes \sigma_i
 \end{align}
 where $p_i = \text{tr}(E_i \rho_{RA}), \omega_i = \frac{1}{p_i} \text{tr}_B \left[(\mathbb{I}_R \otimes \sqrt{E_i})\rho_{RA} (\mathbb{I}_A \otimes \sqrt{E_i})\right]\geq 0$. 
\end{proof}

We note that there are \textit{some} channel capacities of entanglement-breaking channels that are understood.
\begin{itemize}
    \item Quantum information transmission is equivalent to generating entanglement. Because entanglement-breaking channels cannot do that, their quantum capacity is zero.
    \item $C(\mathcal{N}) \geq \mathcal{N}, C(\mathcal{N})=\text{sup}_{n \in \mathbb{N}} \frac{1}{n} \mathcal{X}(\mathcal{N}^{\otimes n})$. Entanglement-breaking channels destroy entanglement between different inputs. This implies that 
    \begin{align}
        C(\mathcal{N}) = \mathcal{X}(\mathcal{N})
    \end{align}
    BUT: $\mathcal{X}(\mathcal{N})$ is NP-hard to compute, so this relationship doesn't actually get us much.
\end{itemize}
\section{PPT-channels}
Checking separability is NP-hard. So, is there some easier criterion? The standard relaxed criterion is following.

\begin{definition}
Peres-Horodecki criterion: if $\rho_{AB}$ is separable if $\rho_{AB}$ has a positive partial transpose with respect to either party.  
\end{definition}
Let us see why this is a reasonable criterion for separability. If $\rho_{AB}$ is separable, 
\begin{align}
    \rho_{AB} &= \sum_i p_i \omega_A^i \otimes \sigma_B^i \\
    \implies  \rho_{AB}^{T_B} &= \sum_i p_i \omega_A^i \otimes (\sigma_B^i)^T \geq 0
\end{align}
\begin{itemize}
    \item if $\rho_{AB}$ is NPT $\implies \rho_{AB} \notin$SEP. 
    \item if $|A|\cdot |B| \leq 6$, then $\rho_{AB} \in$ SEP $\Leftrightarrow \rho_{AB} \in$ PPT.  
\end{itemize}

\begin{definition}
A channel $\mathcal{N}: A \rightarrow B$ is called PPT if ($\mathcal{I}_R \otimes \mathcal{N})(\rho_{RA})$ is PPT for all $\rho_{RA}$.
\end{definition}

\begin{proposition}
The following are all equivalent
\begin{itemize}
    \item $\mathcal{N}: A\rightarrow B$ is PPT
    \item $\tau^{\mathcal{N}}_{AB}$ is PPT 
    \item $\vartheta \circ \mathcal{N}$ is CP (Recall that when $\vartheta: X \mapsto X^T: (\mathbb{I}_R \otimes \vartheta)(\gamma)= \mathbb{F} \ngeq 0$)
\end{itemize}
\end{proposition}

\begin{tcolorbox}[colframe=black,breakable, colback=black!5, arc=0pt, outer arc=0pt,boxrule=0.5pt]
\begin{proof}
(a $\Rightarrow$ b): True by definition. We know
\begin{align}
    (\mathbb{I}_A \otimes \mathcal{N})(\rho_{AA'})
\end{align}
is PPT, in particular for $\rho=\gamma_{AA'}$.

(b $\Rightarrow$ c): \begin{align}
    (\mathbb{I}_A \otimes \vartheta \circ \mathcal{N})(\gamma) &= (\mathbb{I}\otimes \vartheta)(\mathbb{I} \otimes \mathcal{N})(\gamma_{AA'}) \\
    &= (\mathbb{I}_A \otimes \vartheta) (\tau_{AB}^{\mathcal{N}}) \\
    &= (\tau_{AB}^{\mathcal{N}})^{T_B} \geq 0 
\end{align}

(c $\Rightarrow$ a) $\vartheta \circ \mathcal{N}$ is CP: 
\begin{align}
    (\mathbb{I}_R \otimes \vartheta \circ \mathcal{N})(\rho_{RA}) \geq 0 \quad \forall \rho_{RA} \geq 0
\end{align}
This implies $\mathcal{N}$ is PPT.
\end{proof}
\end{tcolorbox}

\textbf{Remarks}: In general, $\vartheta \circ \mathcal{N}$ is \textit{not} CP. However, for every CP map $\mathcal{N}$, the map $\vartheta \circ \mathcal{N} \circ \vartheta$ is CP (see exercises).

What about the capacities of PPT channels?
\begin{itemize}
    \item Horodeckis: PPT states are undistillable. Entanglement distillation: given iid copies of a state $\rho_{AB}$, the goal is to convert these copies into a smaller number of maximally entangled states. If there is an entanglement distillation protocol (LOCC) such that 
    \begin{enumerate}
        \item error of the protocol tends to 0 as $n\Rightarrow \infty$
        \item rate c:= $\frac{1}{n} \log{m_n} \rightarrow c>0$ as $n$ goes to infinity
    \end{enumerate}
    then $\rho_{AB}$ is \textit{distillable}. We conclude that the quantum capacity for PPT channels is zero. This result even holds if two-way classical communication is allowed!
\end{itemize}

We note that the set of protocols allowing only one-way communication is a strict subset of the set of protocols allowing two-way communication. Even still, the Horodecki's showed that the quantum capacity of all PPT channels is zero in both cases. Also note that the classical capacity is generally unknown. 

\section{Anti-degradable channels}
Let $\mathcal{N}$ be a quantum channel from $A$ to $B$ with an isometry $V: \mathcal{H}_A \rightarrow \mathcal{H}_B \otimes \mathcal{H}_E$ such that $\mathcal{N}(\rho) = \text{tr}_E V \rho V^{\dagger}$. Recall that the completmentary channel is $\mathcal{N}^c (\rho) = \text{tr}_B V \rho V^{\dagger}$.

\begin{definition}
$\mathcal{N}$ as above is called antidegradable if there exists a channel $\mathcal{A}: E \rightarrow B$ such that 
\begin{align}
    \mathcal{N} = \mathcal{A}\circ \mathcal{N}^c
\end{align}
\end{definition}
Intuition: Eve (environment) can locally obtain Bob's output via the channel $\mathcal{A}$. Antidegradable channels cannot transmit quantum information and thus have zero quantum capacity, $Q(\mathcal{N})=0.$

A ``proof'' of this is as follows. Assume that this channel has non-zero quantum capacity. This means that Alice can faithfully send qubits to Bob at a positive rate. But there is a protocol based on the channel $\mathcal{A}$ that lets Eve implement the same protocol that Alice and Bob use. This violates no-cloning and is thus prohibited. 

\noindent \textbf{Examples of anti-degradable channels}:
\begin{enumerate}
    \item erasure channel $\mathcal{E}_p: \rho \mapsto (1-p)\rho + p \text{tr}(\rho) \ket{e}\bra{e}$ for $p\geq \frac{1}{2}$. 
    
    Let $\mathcal{H}_1 = \mathbb{C}^2$ be the input space. Let the $\mathcal{H}_2=\mathbb{C}$ be an erasure flag. Let $\mathcal{E}_p : B(\mathcal{H}_1) \rightarrow B(\mathcal{H}_1 \oplus \mathcal{H}_2)$. That is, an input state maps as
    \begin{align}
        \rho \mapsto (1-p) \tilde{\rho} + p \text{tr}(\rho) \ket{e}\bra{e}
    \end{align}
    So, how do we embed our input state in the larger space? We use a pretty trivial embedding that is represented as:
    \begin{align}
        \hat{\rho} = \begin{pmatrix}
        \rho_{00} & \rho_{01} & 0 \\
        \rho_{10} & \rho_{11} & 0 \\
        0 & 0 & 0 
        \end{pmatrix}
    \end{align}
    and the erasure flag is simply
    \begin{align}
        \ket{e}\bra{e} =  \begin{pmatrix}
        0 & 0 & 0 \\
        0 & 0 & 0 \\
        0 & 0 & 1
        \end{pmatrix}.
    \end{align}
    Note that the complementary channel of the erasure channel has the same form but with the probabilities flipped:
    \begin{align}
        \mathcal{E}^c_p: B(\mathcal{H}_1) \rightarrow B(\mathcal{H}_1 \oplus \mathcal{H}_2 ), \rho \mapsto p \tilde{\rho} + (1-p) \text{tr}(\rho) \ket{e}\bra{e}.
    \end{align}
    Now consider $p\geq \frac{1}{2}$ and define $q=\frac{2p-1}{p}$. the idea is that we erase $\tilde{\rho}$ with probability $q$ and do nothing with $\ket{e}\bra{e}$.
    
    \begin{align}
        \mathcal{E}_q (p\rho) &= p(1-q) \tilde{\rho} + p q \text{tr}(\rho) \ket{e}\bra{e} \\
        &= (1-p) \tilde{\rho} + (2p-1) \ket{e}\bra{e} \\
        &= (1-p) \ket{e}\bra{e}
    \end{align}
    \textbf{almost certainly messed up p's and q's here.}
    
    We need to extend the action of $\mathcal{E}_q$ to $B(\mathcal{H}_1 \otimes \mathcal{H}_2)$. The solution is to define $\mathcal{A}$ in the Kraus representation. The Kraus operators are 
    \begin{align}
       K_0 &= \sqrt{1-q}\begin{pmatrix}
        1 & 0 & 0\\
        0 & 1 & 0 \\
        0 & 0 & 0
        \end{pmatrix} \\
        K_1 &= \sqrt{q}\begin{pmatrix}
        0 & 0 & 0\\
        0 & 0 & 0 \\
        1 & 0 & 0
        \end{pmatrix} \\
        K_2 &= \sqrt{q}\begin{pmatrix}
        0 & 0 & 0\\
        0 & 0 & 0 \\
        0 & 1 & 0
        \end{pmatrix} \\
        K_3 &= \sqrt{q}\begin{pmatrix}
        0 & 0 & 0\\
        0 & 0 & 0 \\
        0 & 0 & 1
        \end{pmatrix}
    \end{align}
    where the first three operators correspond to erasure on $B(\mathcal{H}_1 \oplus 0)$ and the last to doing nothing on $B(0 \oplus \mathcal{H}_2)$. This definition ensures that $\mathcal{E}_p = \mathcal{A} \circ \mathcal{E}_p^c $ for $p \geq \frac{1}{2}$.  We have $B(\mathcal{H}_1 \oplus 0), B(0\oplus \mathcal{H}_2): \mathcal{A}_p = \mathcal{E}_p \oplus \mathbb{I}$.
    
    For $\rho \in B(\mathcal{H}_1),$ $\sigma = (1-\lambda) \tilde{\rho} + \lambda \ket{e}\bra{e}$. This has the block matrix:
    \begin{align}
        \begin{pmatrix}
        (1-\lambda)\rho & 0 \\
        0 & \lambda
        \end{pmatrix}
    \end{align}
    So 
    \begin{align}
        (\mathcal{E}_p \oplus \mathbb{I})(\sigma) = \begin{pmatrix}
        \mathcal{E}_p((1-\lambda)\rho) & 0 \\
        0 & \mathbb{I}(\lambda)
        \end{pmatrix}
    \end{align}
    
    Then for $\omega \in B(\mathcal{H}_1 \oplus \mathcal{H}_2)$ we have 
    \begin{align}
        \omega = \begin{pmatrix}
        \omega_{\mathcal{H}_1} & * \\
        * & \omega_{\mathcal{H}_2}
        \end{pmatrix}
    \end{align}
    which has off-diagonal elements. We can get rid of these by measuring with respect to $\mathcal{H}_1 \oplus \mathcal{H}_2 = \mathcal{H}$. Let $P_2 = \ket{e}\bra{e}, P_1 = \mathbb{I}-\ket{e}\bra{e}$. Then $P_1 \omega P_1 + P_2 \omega P_2$ is diagonal. Finally, $\mathcal{A}_M = P_1 \cdot P_1 + P_2 \cdot P_2$ which implies $\mathcal{A}- \mathcal{A}_p \circ \mathcal{A}_M$ where $\mathcal{A}_p = \mathcal{E}_p \oplus \mathbb{I}$. 
    \item amplitude damping channel $\mathcal{A}_{\gamma}$ for $\gamma \geq \frac{1}{2}$. 
    \item depolarizing channel $\mathcal{D}_p$ for $p \geq \frac{1}{4}$
\end{enumerate}

\begin{definition}
Let $\mathcal{N}: A \rightarrow B$ be a quantum channel and let $G$ be a group with unitary representations $U_{g}$ on $\mathcal{H}_A$ and $V_{g}$ on $\mathcal{H}_B$. Then $\mathcal{N}$ is called covariant with respect to $(G,U_g,V_g)$ if 
\begin{align}
    V_{g} \mathcal{N}(\cdot) V_g^{\dagger} = \mathcal{N}(U_g \cdot U_g^{\dagger})
\end{align}
for all $g\in G$.
\end{definition}

\noindent \textbf{Representation theory basics:}
\begin{itemize}
    \item $(\varphi,V)$ is a representation of $G$: $\varphi: G \rightarrow GL(V), \varphi(gh)=\varphi(g) \varphi(h)$. Subspace $W\leq V$ is called $G$-invariant, if $\varphi (g) w \in W \quad \forall w \in W, \forall g \in G$.
    \item $\{0\},V$ are always $G$-invariant
    \item $(\varphi, V)$ is called irreducible if $\{0\},V$ are the only $G$-inv subspaces.
    \item Schur's lemma: $(\varphi,V), (\psi,W)$ representations of $G$. $G$-linear map $f: f \circ \varphi(g)=\psi(g) \circ f$. If $\varphi,\psi$ are irreducible representations, then either $V \ncong W$ and $f=0$ or $V\cong W$ and $f=\lambda \mathbb{I}_{V\rightarrow W}$ for some $\lambda \in \mathbb{C}.$
\end{itemize}

\noindent \textbf{Some examples:}
\begin{itemize}
    \item Pauli channels: $\rho \mapsto p_0 \rho + p_1 X\rho X + p_2 Y\rho Y+p_3 Z \rho Z$
    
    Covariance group: Pauli group, $P=\{\pm 1, \pm i\} \bigcup \{\mathbb{I},X,Y,Z\}$
    
    For $\theta_1,\vartheta_2 \in P$: $\vartheta_1 \theta_2 \cdot \theta_2^{\dagger} \vartheta_1^{\dagger} =\vartheta_2 \theta_1 \cdot \theta_1^{\dagger} \vartheta_2^{\dagger}$ 
    
    \item depolarizing channel: $\rho \mapsto (1-p)\rho + \frac{p}{3} (X\rho X + Y\rho Y + Z \rho Z)$. The covariance group is $\mathcal{U}(2)$. This is easy to see in the ''$q$-representation": $\rho \mapsto (1-q)\rho + q \text{tr}\rho \frac{1}{2} \mathbb{I}$
    
    \begin{align}
        \mathcal{D}_q(U \rho U^{\dagger}) = (1-q) U\rho U^{\dagger} + q \text{tr}(U\rho U^{\dagger}) \frac{1}{2}\mathbb{I} = U \mathcal{D}_p (\rho) U^{\dagger}
    \end{align}
    
    In $d\geq$ dimensions: $\rho \mapsto (1-q) \rho + q \text{tr}\rho \frac{1}{d} \mathbb{I}$. The covariance group is $\mathcal{U}(d)$. 
    
    \item Amplitude damping channel $A_{\gamma} = \{K_0, K_1\}$ as normal. Covariance group: $\{\mathbb{I},Z\} \cong \mathbb{Z}_2$.
    
    \item Erasure channel $\mathcal{E}_p (\rho) = (1-p)\rho + p \text{tr}\rho \ket{e}\bra{e}$. The covariance group is again $\mathcal{U}(2)$. Similarly, in $d\geq 2$ dimensions, the $d$-dimensional erasure channel with covariance group $\mathcal{U}(d)$.
\end{itemize}

\begin{proposition}
A channel $\mathcal{N}: A \rightarrow B$ is $(G,U_g,V_g)$-covariant iff $\tau^{\mathcal{N}_{AB}}=(\bar{U}_g \otimes V_g) \tau^{\mathcal{N}}_{AB} (\bar{U}_g \otimes V_g)^{\dagger}$ for all $g \in G$.
\end{proposition}

\begin{proof}
($\Rightarrow$): $V_g \mathcal{N} (\cdot) V_g^{\dagger} = \mathcal{N}(U_g \cdot U_g^{\dagger})$ for all $g\in G$. This is equivalent to $\mathcal{N}(\cdot) = V^{\dagger}_g \mathcal{N}(U_g \cdot U^{\dagger}_g)V_g$ for all $g\in G$.

\begin{align}
    \tau^{\mathcal{N}}_{AB} &= (\mathbb{I} \otimes \mathcal{N})(\gamma)\\
    &= (\mathbb{I}_1 \otimes V_{g}^{\dagger})(\mathbb{I} \otimes \mathcal{N})\left((\mathbb{I}_1 \otimes U_g)\gamma(\mathbb{I} \otimes U^{\dagger}_g)\right)(\mathbb{I}_1 \otimes V_g) \\
    &= (\mathbb{I} \otimes V_g)(\mathbb{I} \otimes \mathcal{N}) \left( (U_g^{T} \otimes \mathbb{I}_2) \gamma (U^T_g \otimes \mathbb{I}_2)^{\dagger} \right)(\mathbb{I} \otimes V_g) \\
    &= (U^T_g \otimes V^{\dagger}_g)(\mathbb{I}\otimes \mathcal{N})(\gamma)(\bar{U}_g \otimes V_g)\\
    \implies & \tau^{\mathcal{N}}_{AB}= (U^T_g \otimes V^{\dagger}_g) \tau^{\mathcal{N}}_{AB} (\bar{U}_g \otimes V_g) \quad \text{ for all} \quad g \in G
\end{align}

($\Leftarrow$): $\tau^{\mathcal{N}}_{AB} = (\bar{U}_g \otimes V_g) \tau^{\mathcal{N}}_{AB} (\bar{U}_g \otimes V_g)^{\dagger}$ for all $g\in G$. This implies $(U_g^T \otimes \mathbb{I}) \tau^{\mathcal{N}}_{AB} (U_g^T \otimes \mathbb{I})^{\dagger} = (\mathbb{I}\otimes V_g)\tau^{\mathcal{N}}_{AB}(\mathbb{I} \otimes V_g)^{\dagger}$.


\noindent Recall the choi isomorphism: $\mathcal{N}(X) = \text{tr}_1 \tau^{\mathcal{N}}_{AB} (X^T \otimes \mathbb{I})$.

\begin{align}
    \mathcal{N}(U_g X U_g^{\dagger})&=\text{tr}_1 \left[\tau^{\mathcal{N}}_{AB}((U_g X U_g^{\dagger})^T \otimes \mathbb{I})\right]\\
    &= \text{tr}_1 \left[(U_g^T \otimes \mathbb{I})\tau_{AB} (U_g^T \otimes \mathbb{I})^{\dagger}(X^T \otimes \mathbb{I})\right]\\
    &= \text{tr}_1 \left[(\mathbb{I}\otimes V_g)\tau_{AB} (\mathbb{I} \otimes V_g)^{\dagger} (X^T \otimes \mathbb{I})\right]\\
    &=V_g \text{tr}_1 \left[\tau^{\mathcal{N}_{AB}} (X^T \otimes \mathbb{I}) \right]V_g^{\dagger}\\
    &= V_g \mathcal{N}(X)V_g^{\dagger}
\end{align}
\end{proof}

\noindent \textbf{Problem:} with the above proposition is that it is basis-dependent (via the Choi operator). 

\noindent\textbf{Solution:} is that there is also a basis independent version based on the Jamiolkowski operator $J^{\mathcal{N}}_{AB} = (\mathbb{I}\otimes \mathcal{N})(\mathbb{F}_{AA'})$ where $\mathbb{F}_{AB} =\ket{\gamma}\bra{\gamma}^{T_A}$. In terms of the Jamialkowski operator, $(G,U_g,V_g)$-covariance of $\mathbb{N}$ is equivalent to $(U_g \otimes V_g) J^{\mathcal{N}}_{AB} (U_g \otimes V_g)^{\dagger} = J_{AB}^{\mathcal{N}}$ for all $g \in G$.

We note that a $d$-dimensional depolarizing channel; $\rho \mapsto (1-q)\rho + q \text{tr}\rho \frac{1}{d} \mathbb{I}_d$ has $\mathcal{U}(d)$ as its covariance group. 

\begin{proposition}
Let $\mathcal{N}: A\rightarrow B$ be a channel with input and output spaces of equal dimension $d$. If $U \mathcal{N}(\cdot) U^{\dagger}=\mathcal{N}(U \cdot U^{\dagger})$ for all $U\in mathcal{U}(d)$, then 
\begin{align}
    \mathcal{N}=(1-q) \cdot + q \text{tr}(\cdot) \frac{1}{d} \mathbb{I}_d
\end{align}
with $q=(1-f)/(1-d)$ where $f=\bra{\gamma}\tau^{\mathcal{N}}\ket{\gamma}$.
\end{proposition}
\bibliography{references}{}
\bibliographystyle{unsrt}


\end{document}
