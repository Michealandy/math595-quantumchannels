\documentclass[../../note.tex]{subfiles}

\begin{document}

\chapter{Introduction}
I collect all the tools here. I have two approaches to collect these tools, i.e., the bottom-top and top-bottom approach. 

The bottom-top one means that I collect it from basic materials such as textbooks or courses. For example, I learn the linear programming in the optimization textbook. Then in the research side, I may use linear programming to sovle research problems. In a word, I first have a knife and then use it to cut things. This is different from the textbook learning. In textbook learning, I learn general knowledge and do not konw what applications they have. Only after they have applications, I will call them tools. So generally, I put the notes of textbook or course learning in other parts. In those parts, my aim is to build a whole framework of those theories. But in this part, I aim to learn deeper for a particular technique.

The top-bottom one means that I extract it directly from the papers. This means I learn a protocol from a paper to solve a particular problem. Then I extract the fundamental technique from the protocol. This technique will be in my toolbox and I will use it to solve other problems. This approach is more efficient espeically for begineers.

\end{document}